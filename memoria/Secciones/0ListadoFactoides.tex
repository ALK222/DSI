\chapterANonNumber{Listado de factoides}
El listado de factoides es el producto resultado de la fase de investigación. Para obtenerlo se ha extraído las principales características de las
diversas entrevistas, de todas las respuestas a los cuestionarios y del análisis de la competencia. Respecto a los obtenidos en el anterior hito,
hemos realizado una serie de cambios debido a la falta de respuestas a algunas de las variables que hemos identificado y a los problemas que se
han podido identificar durante la corrección de dichos factoides.

\textbf{Factoides de Madi:}

\begin{itemize}
    \item Madi es secretaria de la FEDDI y lleva 16 años trabajando allí.
    \item Madi se encarga de organizar los viajes cuadrando horarios y comprando billetes.
    \item Madi compra billetes a través de Renfe, Air Europa o Iberia porque le sale más económico que en un comparador.
    \item Madi se encarga de los viajes de campeonatos internacionales que les recepciona, les recoge y les lleva al punto de encuentro. Los nacionales los llevan los clubes.
    \item Madi ve que el problema principal para los deportistas es la dependencia de sus padres y entrenadores, el manejo de las aplicaciones y el internet y para algunos necesitan acompañante.
    \item Madi considera que necesitan acompañantes porque tienen problemas para orientarse y no se manejan bien.
    \item Para Madi las dificultades dependen mucho del grado de discapacidad.
    \item Madi no usa aplicaciones, sólo compra en páginas oficiales de la aerolínea.
    \item Madi utiliza la agencia de viajes Betravel para vuelos internacionales.
    \item Madi descarta Ryanair.
    \item Madi no utiliza comparadores porque ya tiene localizadas dos compañías y Renfe que ofrecen el servicio de acompañamiento de AENA.
    \item Madi tiene en cuenta el precio y la variedad de horarios para coger un billete.
    \item Madi no se fija en más facilidades, esos son los atletas.
    \item Madi considera que el comparador debe ser muy simple (lugar, destino y fecha).
    \item A Madi le parece importante que se pueda ver qué tipos de servicios de acompañantes ofrecen.
    \item Madi comenta que hay bastantes campeonatos.
    \item En los campeonatos de España son los clubes los encargados del desplazamiento de los deportistas (ella no interviene).
\end{itemize}


\textbf{Factoides de Sofia:}

\begin{itemize}
    \item Sofía tiene 21 años y es estudiante.
    \item A Sofía le gusta viajar y desde pequeña ha querido conocer las diferentes partes del mundo.
    \item Sofía viaja a menudo, tanto fuera como dentro de España.
    \item A Sofía le encantaría poder viajar más.
    \item Sofía usa tanto automóviles como trenes y autobuses en sus viajes dependiendo del sitio al que viaje.
    \item Sofía prefiere usar autobuses solo cuando viaje distancias cortas o medias.
    \item Sofía ha hecho viajes de varios tipos. Desde intercambios lingüísticos a viajes familiares o con amigos, para conocer nuevas ciudades o relajarse.
    \item A Sofía le gusta ir alternando entre viajar sola, con familia o con amigos, disfruta de todas.
    \item Sofía no recuerda haber tenido ningún problema viajando, aunque admite que cuando viaja lo hace con la mente más abierta de lo que la tiene normalmente.
    \item Sofía a veces organiza los viajes que hace y a veces no.
    \item Sofía busca viajes económicamente accesibles.
    \item Sofía utiliza varios comparadores de viajes a la hora de organizar un viaje. Por ejemplo: Kayak, Skyscanner, Trivago.
    \item A Sofía le gustaría que los comparadores de viajes mostraran el precio final del billete, con los posibles extras incluidos, piensa que es un punto a mejorar.
    \item Sofía prefiere usar aplicaciones web a aplicaciones móviles, y suele hacer estas gestiones desde el ordenador.
    \item Sofía piensa que algunos comparadores son tediosos a la hora de usar, ya que algunos tardan mucho en cargar o te redirigen a otras páginas.
    \item Cuando le ha ocurrido esto, Sofía ha optado por usar otro comparador que no tuviera estos problemas.
    \item Como aportación, a Sofía le gustaría que los comparadores incluyeran el precio final de los billetes, desglosados con los diferentes conceptos.
    \item Sofía considera que los comparadores de viajes son bastante accesibles, pero que quizás aclarar algunas cosas en las webs o los anuncios de spam en las webs pueden molestar a personas con discapacidad intelectual.
    \item Sofía considera que Whatsapp es una aplicación accesible.
    \item Sofía considera que tiene un buen manejo de la tecnología.
\end{itemize}


\textbf{Factoides de Alberto:}

\begin{itemize}
    \item Alberto tiene 22 años y es informático.
    \item Alberto no tiene discapacidad.
    \item Alberto se desenvuelve bien con las tecnologías y le parecen fáciles de usar.
    \item A Alberto le gusta viajar para descubrir historias, paisajes y nuevas culturas.
    \item Alberto viaja exclusivamente por ocio.
    \item Alberto viaja una vez al mes.
    \item A Alberto le gustaría viajar más, más adelante fuera de España (por ahora prefiere conocer la península).
    \item Alberto usa normalmente vehículo propio para viajar.
    \item Alberto hace 7 años que no viaja en avión.
    \item Alberto usa ocasionalmente taxis, pero prefiere el transporte público.
    \item Alberto descarta el uso de Uber.
    \item Alberto suele viajar con su pareja.
    \item Alberto prepara o busca un itinerario antes de viajar, incluso toma apuntes de paradas por si le falla el móvil o gps, le parece tedioso igualmente y se encarga con su pareja.
    \item Alberto sólo usa comparadores para alojamientos mirando lo visual que sea, la localización y el precio.
    \item Alberto usa Booking y le parece incómodo que le obliguen a poner fechas para ver el precio, prefiere ver el precio directamente.
    \item A Alberto le gusta ver pocas ofertas y más relevantes, y no ver muchas ofertas.
\end{itemize}


\textbf{Factoides de Beatriz:}
\begin{itemize}
    \item Beatriz tiene 21 años.
    \item Beatriz se desenvuelve bien con las tecnologías y declara usarlas a diario.
    \item A Beatriz le gusta viajar para conocer otros lugares y culturas.
    \item Beatriz viaja una vez al año, sobre todo dentro de España, y fuera de España una vez cada dos años.
    \item A Beatriz le gustaría viajar más; son razones económicas y de tiempo lo que se lo impide.
    \item Suele viajar en coche o en avión para distancias más largas.
    \item Beatriz suele viajar para visitar a familiares o por razones de ocio.
    \item Beatriz suele viajar acompañada (normalmente de sus familiares).
    \item Cuando viaja con su familia, organizan el viaje conjuntamente.
    \item Cuando no viaja con su familia, Beatriz suele organizar los viajes que hace.
    \item Beatriz se fija sobretodo en el precio a la hora de tomar una decisión, por ejemplo, para comprar un vuelo.
    \item En su último viaje, Beatriz optó por usar una aplicación (eDreams) debido a que tenían un programa con prueba gratuita con el que sus vuelos le salieron más baratos.
    \item Beatriz prefiere los vuelos de ida tempranos y los de vuelta más tarde.
    \item A la hora de hacer la reserva, Beatriz cree que lo más tedioso son todas las pantallas que tienes que atravesar en las que te ofrecen todo tipo de servicios extra, alguno incluso ofreciéndose dos veces.
    \item A Beatriz también le gustaría que los comparadores pusieran un mensaje más claro en el caso de que ida y vuelta sean desde aeropuertos distintos.
    \item Las aplicaciones que más usa Beatriz son Chrome, Whatsapp y Discord.
    \item Google es su navegador favorito, principalmente porque lo lleva usando mucho tiempo y está acostumbrada, pero también debido a la conectividad con otros servicios en su móvil y porque considera que es lo mejor en cuanto a desarrollo web; también le gusta su estética.
    \item El comparador favorito de vuelos de Beatriz es el comparador de Google.
    \item Beatriz considera que los comparadores de vuelos pueden no ser accesibles para personas que no tengan mucha soltura con las tecnologías.
    \item Beatriz considera que eDreams es muy estético, más que el comparador de Google o que otros.
    \item Como aportación, a Beatriz le gustaría que los comparadores te sugirieran vuelos desde sitios cercanos a los que estás buscando, sobretodo si están son más económicos.
    \item Para ella no es un problema sacrificar facilidades como el tipo y la cantidad de equipaje, la elección de asiento o los horarios de ida y vuelta para que el precio sea más barato.
\end{itemize}


\textbf{Factoides del cuestionario:}

\begin{itemize}
    \item La mitad de los usuarios tienen entre 19 y 25 años y la otra mitad entre 26 y 65 años.
    \item La mayoría de los encuestados viven en la ciudad, pocos en pueblos.
    \item Dos tercios de los encuestados tienen un poder adquisitivo medio. Un tercio, bajo y muy pocos, alto.
    \item La mayoría de los encuestados les gusta viajar, pocos no.
    \item La mayoría de los encuestados le gusta viajar por conocer nuevos lugares y los pocos que no, es por la gente o por no parar de moverse.
    \item La mayoría de los encuestados viajan al menos una vez al año, el resto viajan al menos una vez al mes y muy poco no viajan.
    \item La mayoría de los encuestados le gustaría viajar más, el resto no.
    \item Todos los encuestados disfrutan cuando viajan.
    \item Los medios de transporte que usan los encuestados son coche propio, transporte público y aéreo.
    \item Los encuestados suelen viajar por ocio, pocos por trabajo.
    \item Las herramientas que más utilizan los encuestados son Trivago, Kayak y SkyScanner. Hay un tercio de los encuestados que no usan ninguna.
    \item Los comparadores de viajes (Trivago, Kayak, Rastreator, SkyScanner, Momondo) son de fácil uso.
    \item Un quinto de los encuestados tienen alguna discapacidad reconocida.
    \item Los encuestados con discapacidad la mayoría tienen discapacidad física y el resto es intelectual o mental.
    \item Los encuestados con discapacidad dos tercios necesitan adaptaciones para sus viajes como para sillas de ruedas.
    \item La mayoría de los encuestados con discapacidad a veces planifican y el resto o nunca o siempre.
    \item Un tercio de los encuestados con discapacidad encuentran dificultad en el proceso de búsqueda por la accesibilidad.
    \item La mayoría de los encuestados con discapacidad no les supone una dificultad buscar medio de transporte, hacer, comparar y ver las ventajas/desventajas de las rutas o comparar precios.
    \item La mitad de los encuestados con discapacidad no usan comparadores de viajes o similares.
    \item Los encuestados con discapacidad que han usado comparadores la mayoría ha desistido.
    \item Para los encuestados con discapacidad les parece complejo solicitar ayuda dentro de la aplicación de viajes.
    \item La inmensa mayoría de los encuestados sin discapacidad ha viajado en el último año.
    \item La mayoría de los encuestados sin discapacidad hace búsqueda de viaje.
    \item Dos tercio de los encuestados sin discapacidad han utilizado un comparador de viajes.
    \item El motivo principal de los encuestados sin discapacidad es ahorrar dinero. También está mayor oferta, facilidad de uso y ahorra tiempo.
    \item Los encuestados sin discapacidad no tienen problemas con los comparadores.
    \item A los encuestados sin discapacidad les supone una dificultad en el tema de la accesibilidad tener que hacer muchas operaciones para llegar a un objetivo.
    \item Los encuestados están de acuerdo en su mayoría de que está bien la accesibilidad salvo por la ausencia de ayudas al rellenar.
    \item La mayoría de los encuestados no ha echado en falta ninguna funcionalidad.
\end{itemize}


\textbf{Factoides del análisis de competencia}

\begin{itemize}
    \item Las funcionalidades buscadas en otras aplicaciones son búsqueda de alojamiento, búsqueda de medio de transporte, comparar precios para el mismo viaje y comprar billetes y alojamientos.
    \item La competencia total a la aplicación está formada principalmente por kayak, eDreams, Momondo y SkyScanner.
    \item La competencia parcial a la aplicación está formada por Trivago, Iberia y Booking.
    \item La mayor diferencia entre Trivago/Booking y nuestra aplicación es que Trivago y Booking son aplicaciones exclusivamente de alojamiento.
    \item La mayor diferencia entre Iberia y nuestra aplicación es que Iberia solo ofrece búsqueda de vuelos de Iberia y no permite buscar alojamiento u otros tipos de transporte.
    \item La aplicación eDreams no permite buscar buses ni tiene flexibilidad de fechas.
    \item SkyScanner solo ofrece como modelo de transporte vuelos y no tiene flexibilidad de fechas.
    \item Kayak y Momondo no tienen sistema de puntos ni suscripción prime.
    \item Todas las páginas para comparar transporte tienen un buen sistema de filtración de opciones.
    \item Todas las aplicaciones ofrecen como mínimo poder buscar vuelos, algunas incluyen trenes y buses en las búsquedas.
    \item No todas las aplicaciones permiten buscar con flexibilidad de fechas.
    \item Todas las aplicaciones tienen un sistema de opiniones y de valoración con estrellas.
\end{itemize}