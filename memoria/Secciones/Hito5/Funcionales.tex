\section{Elementos funcionales}

Los elementos funcionales son las acciones que pueden ejercer los usuarios
sobre los datos que hemos definido en el apartado anterior. Los haremos basándonos
en los requisitos que hemos encontrados en los anteriores hitos.

Vamos a incluir todos los elementos funcionales, pero marcaremos aquellos que estén
repetidos con respecto al hito anterior. Todos han sido adaptados para indicar la ventana exacta
en la que ocurren las distintas funcionalidades.

\subsection*{Repetidos}

\begin{itemize}
    \item \textbf{Ofrecer \textit{(acción)} soporte al cliente \textit{(objeto)} para resolver problemas de gestión de
        billetes de manera rápida y eficaz. \textit{(contexto)}.}
        \begin{itemize}
            \item Desde cualquier pantalla se puede acceder a la pantalla de atención al cliente.
        \end{itemize}

    \item \textbf{Buscar \textit{(acción)} transportes disponibles \textit{(objeto)} a las ciudades designadas \textit{(contexto)}.} 
        \begin{itemize}
            \item Desde la página \textit{Inicio} podemos seleccionar origen, destino y fechas en las que se quiera
                realizar el viaje.
        \end{itemize}

    \item \textbf{Filtrar \textit{(acción)} opciones de transporte \textit{(objeto)} específicas para personas con discapacidad
        física \textit{(contexto)}.}
        \begin{itemize}
            \item Desde la página \textit{Comparador} se pueden filtrar todas las opciones de transporte adecuadas para personas con
                discapacidad física.
        \end{itemize}

    \item \textbf{Seleccionar \textit{(acción)} asientos \textit{(objeto)} una vez elegido el transporte \textit{(contexto)}.}
        \begin{itemize}
            \item Desde la pantalla \textit{Reserva} se pueden seleccionar los asientos deseados.
        \end{itemize}

    \item \textbf{Ofrecer \textit{(acción)} asientos \textit{(objeto)} una vez elegido el transporte \textit{(contexto)}.}
        \begin{itemize}
            \item Desde la pantalla \textit{Reserva} se pueden seleccionar los asientos deseados.
        \end{itemize}

    \item \textbf{Seleccionar \textit{(acción)} fechas concretas o un intervalo de tiempo \textit{(objeto)} para la búsqueda del transporte \textit{(contexto)}.}
        \begin{itemize}
            \item Desde la pantalla \textit{Inicio} podemos seleccionar las fechas en las que se quiera realizar el viaje con un abanico de días para
                ofrecer flexibilidad.
        \end{itemize}
    
    \item \textbf{Comparar \textit{(acción)} precios \textit{(objeto)} de los diferentes transportes a la ciudad designada \textit{(contexto)}.}
        \begin{itemize}
            \item Desde la página \textit{Comparador} se puede visualizar todas las opciones de transporte a la ciudad destino.
        \end{itemize}

    \item \textbf{Ofrecer \textit{(acción)} información sobre horarios de transporte \textit{(objeto)} al realizar la búsqueda \textit{(contexto)}.}
        \begin{itemize}
            \item Desde la página \textit{Comparador} se muestran todos los viajes tanto con los horarios de salida como de llegada .
        \end{itemize}

    \item \textbf{Ofrecer \textit{(acción)} información de los asientos disponibles \textit{(objeto)} del vehículo seleccionado \textit{(contexto)}.}
        \begin{itemize}
            \item Desde la página \textit{Reserva} se puede ver cuántos asientos quedan disponibles y cuáles.
        \end{itemize}

    \item \textbf{Poder cancelar \textit{(acción)} una reserva \textit{(objeto)} en caso de que el usuario lo considere pertinente \textit{(contexto)}.}
        \begin{itemize}
            \item En la página \textit{Mis reservas} habrá una opción de cancelar la reserva.
        \end{itemize}

    \item \textbf{Poder modificar \textit{(acción)} una reserva \textit{(objeto)} en caso de que el usuario lo considere pertinente \textit{(contexto)}.}
        \begin{itemize}
            \item En la página \textit{Mis reservas} habrá una opción de modificar la reserva.
        \end{itemize}
\end{itemize}


\subsection*{Nuevos o modificados}

\begin{itemize}
    \item \textbf{Ofrecer \textit{(acción)} información detallada sobre las diferentes empresas que operan
        \textit{(objeto)} al comparar dos viajes \textit{(contexto)}.}
        \begin{itemize}
            \item Desde la página de \textit{Comparador} de viajes se puede hacer click en el logo de la empresa que
                opera para mostrar información e ir a la página del sitio.
        \end{itemize}
    
    \item \textbf{Ofrecer \textit{(acción)} información sobre las ayudas ofrecidas \textit{(objeto)} a usuarios 
        con discapacidad \textit{(contexto)}.}
        \begin{itemize}
            \item Desde la página de \textit{Comparador} se muestra la información sobre las ayudas a personas
                con discapacidad en el viaje.
        \end{itemize}
    
    \item \textbf{Reservar \textit{(acción)} billetes \textit{(objeto)} de los transportes deseados \textit{(contexto)}.} 
        \begin{itemize}
            \item Desde la página \textit{Comparador} se pueden seleccionar los billetes deseados y reservarlos.
            \item Desde la pantalla \textit{Datos adicionales de un viaje} se pueden seleccionar los billetes y reservarlos.
        \end{itemize}

    \item \textbf{Indicar \textit{(acción)} origen y destino \textit{(objeto)} del vehículo a lo largo del trayecto
        \textit{(contexto)}.}
        \begin{itemize}
            \item Desde la pantalla \textit{Reserva} se puede visualizar los detalles de la reserva del transporte, como el origen y el destino (tanto
                la ciudad como el lugar exacto).
            \item Desde la pantalla \textit{Comparador} se puede visualizar los detalles de la reserva del transporte, como el origen y el destino (tanto
            la ciudad como el lugar exacto).
        \end{itemize}

    \item \textbf{Mostrar \textit{(acción)} información detallada \textit{(objeto)} sobre la accesibilidad de los transportes 
        disponibles \textit{(contexto)}.}
        \begin{itemize}
            \item Dentro de la información de cada transporte, accesible en \textit{Comparador} gracias al botón simbolizado con una \textbf{i}, se puede visualizar el nivel de accesibilidad del medio de
                transporte y seleccionar la ayuda si el usuario lo desea.
        \end{itemize}
    
    \item \textbf{Ofrecer \textit{(acción)} servicios disponibles en el transporte \textit{(objeto)} cuando seleccionas un transporte
        en concreto \textit{(contexto)}.}
        \begin{itemize}
            \item Dentro de la página \textit{Comparador} se puede ver los servicios adicionales del transporte gracias al
                botón de información de cada tarjeta (simbolizado con una \textbf{i}).
            \item Dentro de la página \textit{Reserva} se puede ver los servicios adicionales del transporte y reservarlos.
        \end{itemize}
    
    \item \textbf{Realizar \textit{(acción)} reservas \textit{(objeto)} para un número determinado de personas \textit{(contexto)}.}
        \begin{itemize}
            \item Dentro tando de la página \textit{Inicio} como la de \textit{Comparador} se pueden seleccionar cuantos billetes se
                quieren comprar, distinguiendo entre adultos, niños y personas con discapacidad.
        \end{itemize}

    \item \textbf{Filtrar \textit{(acción)} viajes \textit{(objeto)} en función del número de personas que vayan a participar en el viaje \textit{(contexto)}.}
        \begin{itemize}
            \item Al realizar la búsqueda con los billetes con el número de billetes a comprar, se filtra automáticamente para solo mostrar transportes
                que tengan como mínimo ese número de plazas disponibles.
        \end{itemize}

    \item \textbf{Reservar \textit{(acción)} conjuntos de asientos \textit{(objeto)} para que todos los viajeros en caso de que sea un grupo puedan
        sentarse juntos \textit{(contexto)}.}
        \begin{itemize}
            \item Dentro de la pantalla \textit{Reserva}, en el apartado de \textit{Selección de asientos} se pueden seleccionar.
                los asientos deseados.
        \end{itemize}

    \item \textbf{Consultar y elegir \textit{(acción)} la ruta \textit{(objeto)} que más se ajuste a tus necesidades en caso de que haya varias opciones que se puedan seleccionar \textit{(contexto)}.}
        \begin{itemize}
            \item En la página \textit{Comparador} se indicará qué ruta hace cada viaje. De esta manera, el usuario puede elegir la que más se adapte
        \end{itemize}

    \item \textbf{Ofrecer \textit{(acción)} opciones de hacer escalas \textit{(objeto)} en caso de que se quiera hacer un vuelo con estas condiciones \textit{(contexto)}.}
        \begin{itemize}
            \item En la página \textit{Comparador} se pueden filtrar los viajes por si tienen escala/transbordo o no.
        \end{itemize}

    \item \textbf{Filtrar \textit{(acción)} viajes \textit{(objeto)} según precio, duración, tipo de transporte, accesibilidad u horario \textit{(contexto)}.}
        \begin{itemize}
            \item En la página \textit{Comparador} hay una barra con a la derecha con las opciones de filtrado de precio, duración, accesibilidad y horario. Además, encima de los viajes
                de ida y vuelta tenemos la opción de elegir tipo de transporte, mostrando solo los del seleccionado.
        \end{itemize}

    \item \textbf{Ordenar \textit{(acción)} viajes \textit{(objeto)} según precio o duración, para agilizar la búsqueda \textit{(contexto)}.}
        \begin{itemize}
            \item En la página \textit{Comparador}, encima de los viajes de ida y vuelta hay un botón que nos permite seleccionar las distintas ordenaciones.
        \end{itemize}
\end{itemize}