\section{Elementos funcionales}

Los elementos funcionales son las acciones que pueden ejercer los usuarios
sobre los datos que hemos definido en el apartado anterior. Los haremos basándonos
en los requisitos que hemos encontrados en los anteriores hitos.

Vamos a incluir todos los elementos funcionales, pero marcaremos aquellos que estén
repetidos con respecto al hito anterior.

\subsection*{Repetidos}

\begin{itemize}
    \item \textbf{Ofrecer \textit{(acción)} soporte al cliente \textit{(objeto)} para resolver problemas de gestión de
        billetes de manera rápida y eficaz. \textit{(contexto)}.}
        \begin{itemize}
            \item Desde cualquier pantalla se puede acceder a la pantalla de atención al cliente.
        \end{itemize}
\end{itemize}

\subsection*{Nuevos}

\begin{itemize}
    \item \textbf{Ofrecer \textit{(acción)} información detallada sobre las diferentes empresas que operan
        \textit{(objeto)} al comparar dos viajes \textit{(contexto)}.}
        \begin{itemize}
            \item Desde la página de \textit{Comparador} de viajes se puede hacer click en el logo de la empresa que
                opera para mostrar información e ir a la página del sitio.
        \end{itemize}
    
    \item \textbf{Ofrecer \textit{(acción)} información sobre las ayudas ofrecidas \textit{(objeto)} a usuarios 
        con discapacidad \textit{(contexto)}.}
        \begin{itemize}
            \item Desde la página de \textit{Comparador} se muestra la información sobre las ayudas a personas
                con discapacidad en el viaje.
        \end{itemize}

    \item \textbf{Buscar \textit{(acción)} transportes disponibles \textit{(objeto)} a las ciudades designadas \textit{(contexto)}.} 
        \begin{itemize}
            \item Desde la página \textit{Inicio} podemos seleccionar origen, destino y fechas en las que se quiera
                realizar el viaje.
        \end{itemize}

    \item \textbf{Filtrar \textit{(acción)} opciones de transporte \textit{(objeto)} específicas para personas con discapacidad
        física \textit{(contexto)}.}
        \begin{itemize}
            \item Desde la página \textit{Comparador} se pueden filtrar todas las opciones de transporte adecuadas para personas con
                discapacidad física.
        \end{itemize}
    
    \item \textbf{Reservar \textit{(acción)} billetes \textit{(objeto)} de los transportes deseados \textit{(contexto)}.} 
        \begin{itemize}
            \item Desde la página \textit{Comparador} se pueden seleccionar los billetes deseados y reservarlos.
            \item Desde la pantalla \textit{Datos adicionales de un viaje} se pueden seleccionar los billetes y reservarlos.
        \end{itemize}
    
    \item \textbf{Mostrar \textit{(acción)} información detallada \textit{(objeto)} sobre la accesibilidad de los transportes 
        disponibles \textit{(contexto)}.}
        \begin{itemize}
            \item Dentro de la información de cada transporte, se puede visualizar el nivel de accesibilidad del medio de
                transporte y seleccionar la ayuda si el usuario lo desea.
        \end{itemize}
    
    \item \textbf{Seleccionar \textit{(acción)} asientos \textit{(objeto)} una vez elegido el transporte \textit{(contexto)}.}
        \begin{itemize}
            \item Desde la pantalla \textit{Reserva} se pueden seleccionar los asientos deseados.
        \end{itemize}

    \item \textbf{Indicar \textit{(acción)} origen y destino \textit{(objeto)} del vehículo a lo largo del trayecto
        \textit{(contexto)}.}
        \begin{itemize}
            \item Desde la pantalla \textit{Reserva} se puede visualizar los detalles de la reserva del transporte, como el origen y el destino (tanto
                la ciudad como el lugar exacto).
            \item Desde la pantalla \textit{Comparador} se puede visualizar los detalles de la reserva del transporte, como el origen y el destino (tanto
            la ciudad como el lugar exacto).
        \end{itemize}
    
    \item \textbf{Ofrecer \textit{(acción)} asientos \textit{(objeto)} una vez elegido el transporte \textit{(contexto)}.}
        \begin{itemize}
            \item Desde la pantalla \textit{Reserva} se pueden seleccionar los asientos deseados.
        \end{itemize}

\end{itemize}