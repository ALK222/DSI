\section{Introducción}

Tras haber hecho en el hito anterior los primeros bocetos a papel para ver los
distintos escenarios de uso de la aplicación, ahora pasamos a un diseño digital que
representa mejor la visión final del producto. En esta sección recordaremos los puntos
más importantes del hito anterior.

En cuanto a las etapas intercambiables, hemos hecho primero la definición de los \textit{Grupos
funcionales y jerarquías}, tras esto hemos realizado el \textit{Prototipado digital} de la aplicación
(usando \textit{Figma}) y hemos continuado con los \textit{Escenario keypath}. Hemos decidido hacerlo
de esta manera ya que, una vez tenemos los \textit{Elementos funcionales}, hemos pensado que era conveniente
realizar la definición de los grupos para usarlos como base a la hora de diseñar el \textit{Prototipado digital},
así como pensar los principios de diseño del mismo. Tras tener el \textit{Prototipado}, podemos realizar los
\textit{Escenarios keypath} usando el propio prototipo que hemos realizado.