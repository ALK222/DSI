\section{Introducción}
Tras el anterior hito, en el que realizamos el prototipado a papel de nuestra aplicación,
en éste haremos un prototipo digital usando la aplicación \textit{Figma}. 
Para realizar este prototipado seguiremos el mismo proceso dividido en siete etapas que
realizamos en el hito anterior (con posibilidad de una siguiente iteración). Son siete
ya que hemos dividido la definición de elementos de datos y la definición de elementos
funcionales en dos etapas distintas.\\

En estas etapas se va a definir tanto la estructura de alto nivel y la
organización de las pantallas como el flujo, comportamiento y organización del
sistema. Estas etapas son las siguientes y se han realizado de acuerdo al orden
que aparecen descritas a continuación:
\begin{enumerate}
    \item \textbf{Definir el factor de forma, la postura y los métodos de entrada.} Se detalla el contexto en el que se va a mostrar la información, los
    métodos de entrada de la aplicación y la atención que tiene el usuario al interactuar con el sistema.
    \item \textbf{Definir los elementos de datos.} El punto de partida de este apartado van a ser los requisitos identificados en el hito 3 y modificados en el 4.
    En primer lugar, van a definirse los elementos de datos que se van a utilizar en nuestra aplicación.
    \item \textbf{Definir los elementos de funcionales.} Tras definir los elementos de datos, por cada uno de los requisitos se va a explicar cómo lo vamos a integrar en el diseño.
    \item \textbf{Determinar los grupos funcionales y las jerarquías.} En este apartado se va a detallar la jerarquía de las distintas interfaces que van a 
    ser diseñadas para la aplicación, así como el orden en el que se van a usar los elementos y los principios y patrones que se han usado.
    \item \textbf{Hacer un prototipo interactivo digital.} En base a la información obtenida en los pasos anteriores y las anteriores iteraciones (prototipos a papel), hemos realizado
    el prototipo que va a conformar nuestra aplicación de manera interactiva, con el fin de poder interactuar con él para detectar de manera eficiente errores de diseño que pueda
    tener la aplicación.
    \item \textbf{Construir los escenarios keypath} En esta etapa se van a desarrollar los escenarios keypath apoyados en los escenarios de contexto realizados
    en el hito 3. Estos escenarios entran en un mayor detalle del funcionamiento de la aplicación que los escenarios de contexto, por lo que son esenciales para
    un buen prototipado.
    \item \textbf{Validar los diseños con los escenarios de validación.} Por último, como método para validar los prototipos que hemos acordado en conjunto,
    vamos a crear una serie de escenarios de validación que nos van a permitir empatizar con el usuario y poder detectar aquellos fallos y defectos presentes en nuestros
    prototipos.
\end{enumerate}

En cuanto a las etapas intercambiables, hemos hecho primero la definición de los \textit{Grupos
funcionales y jerarquías}, tras esto hemos realizado el \textit{Prototipado digital} de la aplicación
(usando \textit{Figma}) y hemos continuado con los \textit{Escenario keypath}. Hemos decidido hacerlo
de esta manera ya que, una vez tenemos los \textit{Elementos funcionales}, hemos pensado que era conveniente
realizar la definición de los grupos para usarlos como base a la hora de diseñar el \textit{Prototipado digital},
así como pensar los principios de diseño del mismo. Tras tener el \textit{Prototipado}, podemos realizar los
\textit{Escenarios keypath} usando el propio prototipo que hemos realizado.