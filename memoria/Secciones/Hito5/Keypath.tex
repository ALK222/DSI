\section{Escenarios keypath}

Los escenarios keypath son una evolución de los escenarios de contexto que describen detalladamente la interacción entre el usuario y el producto, una técnica de modelado que consiste en describir de manera narrativa cómo utiliza el usuario el producto para lograr sus objetivos. Estos escenarios muestran de manera detallada las funcionalidades del sistema y cómo interactúa el usuario con él.

\subsection{Escenario 1 - Marta}

``Marta quiere ir a las distintas ciudades del norte de Italia de la gira de Francesca, por lo que abre en el ordenador la aplicación e inicia sesión, para en la página de inicio de la aplicación hace clic al icono de usuario y como no está registrada le muestra la pantalla de inicio de sesión, introduce el correo electrónico y su contraseña y le da al botón de iniciar sesión y le redirige a la página de inicio con sesión iniciada.''

``Una vez iniciada la sesión y en la página de inicio, en el buscador selecciona como origen Madrid, como destino la primera ciudad de la gira y las fechas, con el calendario de la página tanto haciendo clic en el calendario de fecha de ida y de vuelta, que habían planificado quedarse en esa ciudad para luego ir a la próxima ciudad según la ruta diseñada y selecciona dos viajeros, ya que va a ir con su hermano y por selecciona un joven y un adulto.''

``Para el primer viaje ha tenido que filtrar los resultados en avión marcando el botón vuelos, ya que tienen que viajar de un país a otro para la primera ciudad y lo más rápido posible. Los resultados aparecen ordenados como predeterminados por fechas, así que no toca la opción de ordenar. Como no tiene ninguna preferencia en filtros no toca la parte izquierda de la página y empieza a navegar primero por los resultados de ida (en la parte izquierda) y selecciona el que le parece mejor y hace lo mismo que los resultados de la vuelta (en la parte derecha) y lo selecciona. Una vez seleccionados ambos viajes le da clic al botón de siguiente.''

``En el siguiente paso, que es la ventana de reserva, Marta rellena sus datos, es decir, pone su nombre, apellidos, DNI, teléfono y correo electrónico en sus campos correspondientes, También selecciona los asientos de ida y vuelta desplegando sus correspondientes menú, siempre elige un asiento asegurándose de que haya al menos un asiento libre para su hermano. Una vez que ha terminado, le da clic al botón de siguiente para hacer lo mismo su hermano. Cuando su hermano ha terminado de rellenar la al botón de siguiente, que le redirige al pago de los billetes.''

``En la página de pago comprueban los asientos mediante haciendo clic en detalles de cada resumen de cada billete y ven que todo está bien. Una vez comprobado esto despliega el menú de tarjeta de crédito y pone el número de la tarjeta, la fecha de caducidad, el nombre del titular y el CVC. Una vez introducida toda la información, presiona pagar y se redirige a la ventana de resumen de pago.''

``En la ventana de resumen de pago miran los dos billetes bien (el de ida y el de vuelta), es decir, comprueban los aeropuertos, el mapa, que la información de los pasajeros estén bien y los asientos estén bien y los descargan y finalmente le dan a seguir explorando porque tienen que comprar más billetes del resto de ciudades del tour.''

Video: ``Marta compra billetes de la primera ciudad del tour''\footnote{Video 1: \url{https://drive.google.com/file/d/1KlXy7L9FnPGZBkQ2OVtC5qDhEe6QtmC0/view?usp=drive_link}}.

``Para el resto de ciudades, hace el mismo procedimiento, a excepción de no poner filtro de medio de transporte en los resultados de la página del comparador, ya que en esos casos da igual mientras cumpla con lo planeado. También cambia que en la ventana de reserva tenga que mirar los servicios adicionales de algunos viajes para contratar el servicio de comida y cama. Por el resto es exactamente igual.''

Video: ``Marta selecciona servicios''\footnote{Video 2: \url{https://drive.google.com/file/d/15AvnvEH5TOCA40rL0Bms_vTr6Ml_wfH8/view?usp=drive_link}}.

``Marta ha tenido que volver a meterse en su perfil a través del icono de usuario en la página de inicio de la aplicación para cancelar un viaje ya que la cantante ha cancelado el concierto en esa ciudad. Para ello le ha dado a la botón con una cruz a la derecha del viaje que quería cancelar y le ha salido un ventana emergente y finalmente ha podido cancelar.''

Video: ``Marta cancela viaje''\footnote{Video 3: \url{https://drive.google.com/file/d/13Jwa1nbN0I86aLVI_jAmkjUNPgPpaHe8/view?usp=drive_link}}.

``Debido a este último cambio, Marta ha tenido que ver si puede adelantar la fecha de ida y cambiar el lugar de origen de la siguiente ciudad a través de comprar un nuevo billete y como si había, compra de la misma forma que antes dos billetes (para su hermano también) con la nueva fecha de ida y cancela el que tenía antes comprado de la misma forma. Como solo quedaban dos asientos ha tenido que rellenar muy rápido los datos en la página de reserva y una vez que ha terminado de pagar, ha empezado a dudar de si había puesto bien los datos. Por lo que desde la página de inicio hace clic en su foto de perfil y le redirige a la ventana de perfil. Ahí hace clic en el botón de mis reservas y ve su lista de reservas. Navega por la página hasta encontrar ese viaje y le da al botón de ver (en forma de ojo) y al comprobar los billetes, efectivamente ha puesto su número y DNI mal. Así que vuelve atrás con el botón de volver y hace clic en el botón de modificar (con el icono de un lápiz) y va al campo de teléfono y dni y lo modifica y le da a confirmar cambios. Y ya está listo su viaje.''

Video: ``Marta modifica sus datos''\footnote{Video 4: \url{https://drive.google.com/file/d/1Nmq3jTpN59hvyDyw6DyOp42AyqlbX7Lo/view?usp=drive_link}}.


\subsection{Escenario 2 - Marta}

``Como dentro de poco son los exámenes y Marta está agobiada, su amiga Pili le ha sugerido ir a Ibiza el fin de semana posterior a los exámenes para relajarse. A Marta le ha parecido bien la idea así que coge su portátil y abre la aplicación con la sesión ya iniciada y en la sección de búsqueda escribe como origen Madrid y como destino Ibiza. Para las fechas las escribe de forma manual en vez de usar el calendario interactivo que son los días del fin de semana acordado. Pone el número de pasajeros a 2 jóvenes y le da al botón de buscar y le redirige a la página de comparador.
Una vez en la página del comparador empieza ordenando los resultados mediante el botón de ordenar y escoger la opción de ordenar por precio de forma creciente pero ve que no son precios muy asequibles, así que aplica el filtro de precio a 50 euros máximo en la parte izquierda y ahora no hay ningún resultado.''

Video: ``Marta filtra resultados''\footnote{Video 5: \url{https://drive.google.com/file/d/1o30b25FNA6LnXUABWE4MGlyzKMH3CAdA/view?usp=drive_link}}.

``Como no podía subir más su presupuesto, vuelve al inicio de la página con el botón de atrás mientras piensa y ve las ofertas que hay al principio y ve que ir a Palma de Mallorca sale 20 euros la ida y vuelta cada uno. Así que después de que Pili cediera al cambio de planes, selecciona en el buscador como origen Madrid, destino Palma de Mallorca, las mismas fechas (igual que antes) y los mismos asientos (2 jóvenes) y pone el filtro de 20€ para que aparezcan antes esos billetes y efectivamente aparecen.''

Video: ``Marta se fija en las ofertas''\footnote{Video 6: \url{https://drive.google.com/file/d/1NEVMNbseElLRKjDC75WIpdf4-KpuRgj-/view?usp=drive_link}}.

``La compra de los billetes es igual que en el primer escenario de Isabel.''


\subsection{Escenario 1 - Isabel}

``Mientras espera el desayuno, Isabel recibe una llamada de su jefe informando sobre la conferencia para la inclusión de niños con discapacidad y la necesidad de encontrar un sustituto para la potente original. Debido a la urgencia, Isabel acepta la propuesta y decide ponerse a buscar transporte para asistir a la conferencia.''

De esta forma, enciende el portátil y busca la aplicación. Una vez abierta, debido a que no tiene una cuenta, decide crearla con la intención de simplificar el proceso para las próximas veces. Para ello, en la página inicial selecciona ``crear cuenta'', lo que la desplaza a una nueva pantalla de registro en la que rellena todos los datos requeridos, nombre, apellidos, fecha de nacimiento, nacionalidad, teléfono, contraseña, entre otros.

Una vez creada su cuenta, se inicia sesión automáticamente y vuelve a la pantalla principal.
Es en esta pantalla donde introduce los datos relativos a su búsqueda de viaje. En el buscador selecciona como origen Barcelona, como destino Madrid, fecha de ida el día actual y fecha de vuelta el miércoles(día en el que tiene la conferencia). Además indica el número de asientos a 1 adulto y marca la casilla de ``viaje accesible''(debido a su condición física) y pulsa en el botón de ``Buscar'' para acceder al comparador.  

Debido a la casilla de ``viaje accesible'' marcada en el buscador, se le despliega la variante del comparador de viajes dedicada a su condición física lo que le permite visualizar de forma más ágil los transportes adecuados. Como el motivo de viaje es urgente, en la sección tipo de transporte selecciona ``vuelos'', para visualizar únicamente los vuelos que cumplan sus condiciones. Además hace uso de los filtros del lateral izquierdo, selecciona viaje directo para evitar transbordos, establece un margen de precio no superior a unos 200 euros y pulsa en el botón ``filtrar'', lo que actualiza la lista de billetes adecuados. Por último utiliza la ordenación por precio inferior en la casilla en la parte superior derecha.

Ya con las opciones disponibles delante, selecciona el billete de ida y vuelta que más le convence y pulsa el botón ``siguiente'' que la lleva al siguiente paso.

En este paso que corresponde a la ventana de reserva Isabel rellena sus datos, nombre, apellidos, DNI, teléfono, etc. Además pulsando en los desplegables, selecciona los asientos de ida y de vuelta disponibles que más le agrade teniendo en cuenta sus limitaciones. Una vez seleccionados los asientos pulsa en ``siguiente'', comenzando así el último paso correspondiente al pago del billete..


En la página de pago comprueba los detalles del billete y selecciona la opción pagar con tarjeta de crédito, introduciendo todos los datos necesarios, número de tarjeta,fecha, titular, cvc, y presiona el botón ``pagar'' para realizar el pago. Es entonces cuando se actualiza la página mostrando un resumen de la compra.

Ya con el billete en su correo, Isabel se siente satisfecha con la utilidad y simplicidad de la aplicación, se dirige a su perfil  para cerrar sesión pulsando en ``cerrar sesión'' y aceptando la confirmación y se dirige a casa a preparar su maleta.

Video: ``Escenario 1 Isabel''\footnote{Video 7: \url{https://drive.google.com/file/d/1zxenSnsUoDwXNdjQV03FPnIg7N1rTsKu/view?usp=drive_link}}.


\subsection{Escenario 2 - Isabel}

``Carmen e Isabel llevan varios meses intentando ir a la exposición de arte de un artista emergente que les gusta mucho. En Barcelona las entradas se agotaron a los pocos minutos de salir, por lo que no lo consiguieron. Ahora ha vuelto con otra exposición, pero esta vez en Madrid.
Como Carmen e Isabel no se la quieren perder, han decidido que viajarán a Madrid un par de días. La exposición estará un mes entero, pero no saben qué días irán.''

Isabel, la cual ya tiene una cuenta creada, abre la aplicación y en la pantalla de inicio introduce los datos relativos a su correo electrónico y contraseña e inicia sesión. Posteriormente ya en la página del buscador, realiza una búsqueda genérica introduciendo como lugar origen Barcelona, lugar destino Madrid y un intervalo de tiempo de 1 mes entero. Para ello marca el comienzo y el final del mes en las fechas correspondientes. Además, quiere realizar la compra de dos billetes, por lo que los compra de forma conjunta indicando como pasajeros, 1 adulto y 1 joven que es su amiga Carmen y pulsa en “buscar”.. 


Por otro lado, Carmen no sufre de discapacidad física, por lo que a la hora de buscar los billetes tendría más ofertas si desactivase la opción de búsqueda personalizada(viaje accesible), por lo que no marcan dicha opción..

Ya en el comparador, busca en concreto los trenes que más se ajustan a sus horarios. Para ello marca la opción ``trenes'' para que les aparezcan únicamente trenes, introducen su horario preferente en el filtro lateral izquierdo y busca a través de las diferentes opciones alguna que cuente con al menos 1 asiento accesible para ella (uso de silla de ruedas). Además, elige billetes de Renfe ya que Carmen tiene descuentos. 

Ya seleccionados los billetes en la página de información adicional, rellena los datos correspondientes para cada uno de los pasajeros(una vez rellenados los datos del pasajero 1 pulsa ``siguiente'' para acceder al siguiente pasajero), nombres, apellidos, DNIs, entre otros y marca la opción ``solicitar asistencia en estación'' para su billete.
Por otro lado tenían la intención de sentarse juntas, por lo que a la hora de elegir el asiento abre y cierra el desplegable para comprobar que efectivamente había seleccionado los asientos correctos.
Una vez introducidos todos los datos, no quedaba más que pagar el billete.

Para ello fue a la pantalla de pago en la que seleccionaron como método de pago ``tarjeta de crédito'', activando el desplegable y rellenando los datos correspondientes al número de la tarjeta,fecha, titular, cvc. Ya con los datos introducidos, le da al botón ``pagar'' realizando la compra de los billetes.

Una vez comprados los billetes, espera un período de tiempo razonable para recibirlos en su correo, pero no llegan. Desde la página, una vez realizado el pago descarga el billete y se da cuenta de que presentan ciertas erratas.

Dada la situación, entra en el apartado de preguntas frecuentes que se encuentra en la parte inferior derecha con el símbolo de interrogación ”?'', para tratar de descubrir la solución a su problema. 
No obstante, al no encontrarla decide ponerse en contacto con el servicio de atención al cliente haciendo clic en el icono que se encuentra sobre el mencionado anteriormente (símbolo en forma de ``chat'').

En el apartado de soporte, selecciona la opción chat, lo que le despliega un chat con un asistente inteligente. Ante sus preguntas, el asistente le dice ``que están teniendo problemas con la gestión de los billetes y que se lo resolverán de forma manual en pocos minutos''. 

Y así fue, instantes después, Isabel ya tenía sus billetes en el correo y su reserva tramitada correctamente.


Video: ``Escenario 2 Isabel''\footnote{Video 8: \url{https://drive.google.com/file/d/1bc_HENHqpWpvK69dTctpTYS1JsREE2tX/view?usp=drive_link}}.
