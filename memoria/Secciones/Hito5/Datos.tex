\section{Elementos de datos}

En esta fase hemos definido los elementos de la interfaz que representarán los
requisitos identificados en la fase de identificación de requisitos.

Hemos creado los elementos de datos (transportes, billetes y viajes), junto con sus
atributos (por ejemplo, horario en el caso de transportes) y las relaciones con otros
elementos de datos, como los transportes y los viajes.

Posteriormente, se ha realizado la traducción de los requisitos funcionales en elementos
funcionales. De forma que describimos las acciones que podemos realizar añadiendo una acción,
un objeto y un contexto.

\subsection{Definición}

Tras un análisis de los errores cometidos en el hito anterior hemos definido los elementos
de la siguiente manera:

\begin{itemize}
    \item \textbf{Viaje}. Con este elemento representamos todos los datos sobre un desplazamiento
    concreto. Contará con los siguientes atributos:
    \begin{itemize}
        \item Fecha. Se marcará tanto el día de salida como el de llegada (pueden ser el mismo),
            con el formato en el orden correspondiente a la región (en España, \textit{dd/mm/aaaa}).
        \item Horario. La hora a la que sale el medio de transporte, además de la hora estimada de 
            llegada. El formato será \textit{HH:MM}, marcando la hora local del origen.
        \item Origen. Ciudad de la que sale el viaje, indicando la estación o aerpuerto correspondiente.
        \item Destino. Igual que con el origen, se guardará la ciudad de destino junto al lugar concreto.
        \item Accesibilidad. Si es un viaje accesible para personas con discapacidad o no.

    \end{itemize}

\end{itemize}
