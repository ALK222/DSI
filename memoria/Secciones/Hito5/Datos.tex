\section{Elementos de datos}

En esta fase hemos definido los elementos de la interfaz que representarán los
requisitos identificados en la fase de identificación de requisitos.

Hemos creado los elementos de datos (transportes, billetes y viajes), junto con sus
atributos (por ejemplo, horario en el caso de transportes) y las relaciones con otros
elementos de datos, como los transportes y los viajes.

Posteriormente, se ha realizado la traducción de los requisitos funcionales en elementos
funcionales. De forma que describimos las acciones que podemos realizar añadiendo una acción,
un objeto y un contexto.

\subsection{Definición}

Tras un análisis de los errores cometidos en el hito anterior hemos definido los elementos
de la siguiente manera:

\begin{itemize}
    \item \textbf{Viaje}. Con este elemento representamos todos los datos sobre un desplazamiento
        concreto. Contará con los siguientes atributos:
        \begin{itemize}
            \item Fecha. Se marcará tanto el día de salida como el de llegada (pueden ser el mismo),
                con el formato en el orden correspondiente a la región (en España, \textit{dd/mm/aaaa}).
            \item Horario. La hora a la que sale el medio de transporte, además de la hora estimada de 
                llegada. El formato será \textit{HH:MM}, marcando la hora local del origen.
            \item Origen. Ciudad de la que sale el viaje, indicando la estación o aerpuerto correspondiente.
            \item Destino. Igual que con el origen, se guardará la ciudad de destino junto al lugar concreto.
            \item Accesibilidad. Si es un viaje accesible para personas con discapacidad o no.
            \item Transporte. El medio utilizado para viajar, en nuestro caso \textit{Avión}, \textit{Tren} o
                \textit{Autobús}.
        \end{itemize}

        Un viaje tendrá asociados tantos billetes como plazas tenga.

    \item \textbf{Billete}. Es el elemento que se paga para tener acceso al \textit{Viaje}. En caso de reservarse asientos
        para varias personas, se considerarán como el mismo \textit{Billete}, con los datos correspondientes de cada comprador.
        Poseerá los atributos siguientes:
        \begin{itemize}
            \item Precio. Cantidad de dinero en la moneda correspondiente que ha pagado el usuario por el \textit{Viaje}.
            \item Usuarios. Datos necesarios para identificar a los viajantes (definido posteriormente).
            \item Asientos. Plazas asignadas a cada uno de los usuarios.
            \item Servicios adicionales. Servicios contratados por uno o varios compradores para utilización durante el
                el \textit{Viaje}.
        \end{itemize}
        Cada \textit{Billete} está asociado a un transporte.

    \item \textbf{Usuario}. Persona que utiliza la aplicación, o que es partícipe de algún \textit{Viaje} (puede ser
        añadida a uno por otro \textit{Usuario}). Contará con los siguientes atributos:
        \begin{itemize}
            \item Nombre. Una cadena de caracteres con el nombre completo de la persona.
            \item Documento de identidad. Una cadena de caracteres correspondiente al documento identificativo de la
                la región pertinente al \textit{Usuario}.
            \item Fecha de nacimiento. Formato usado en la región desde la que se utiliza la aplicación (en España por ejemplo,
                sería \textit{dd/mm/aaaa}).
            \item Correo electrónico. Método para contactar con el \textit{Usuario} en caso de ser necesario (por ejemplo, al
                enviar los billetes, o en caso de que se tenga que notificar la cancelación de un viaje). Será una cadena de
                caracteres que será confirmada a través de un correo electrónico.
        \end{itemize}
        Un usuario tiene uno o varios viajes asociados.

\end{itemize}
