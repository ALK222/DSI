\section{Elementos de datos}

En esta fase hemos definido los elementos de la interfaz que representarán los
requisitos identificados en la fase de identificación de requisitos.

Hemos modificado los elementos definidos en el hito anterior en gran medida, debido a que no
estaban correctamente definidos. Por eso incluimos todos de nuevo.

\subsection{Definición}

Tras un análisis de los errores cometidos en el hito anterior hemos definido los elementos
de la siguiente manera:

\begin{itemize}
    \item \textbf{Viaje.} Con este elemento representamos todos los datos sobre un desplazamiento
        concreto. Contará con los siguientes atributos:
        \begin{itemize}
            \item Fecha. Se marcará tanto el día de salida como el de llegada (pueden ser el mismo),
                con el formato en el orden correspondiente a la región (en España, \textit{dd/mm/aaaa}).
            \item Horario. La hora a la que sale el medio de transporte, además de la hora estimada de 
                llegada. El formato será \textit{HH:MM}, marcando la hora local del origen.
            \item Origen. Ciudad de la que sale el viaje, indicando la estación o aerpuerto correspondiente.
            \item Destino. Igual que con el origen, se guardará la ciudad de destino junto al lugar concreto.
            \item Accesibilidad. Si es un viaje accesible para personas con discapacidad o no.
            \item Transporte. El medio utilizado para viajar, en nuestro caso \textit{Avión}, \textit{Tren} o
                \textit{Autobús}.
        \end{itemize}

        Un viaje tendrá asociados tantos billetes como plazas tenga.

    \item \textbf{Billete.} Es el elemento que se paga para tener acceso al \textit{Viaje}. En caso de reservarse asientos
        para varias personas, se considerarán como el mismo \textit{Billete}, con los datos correspondientes de cada comprador.
        Poseerá los atributos siguientes:
        \begin{itemize}
            \item Precio. Cantidad de dinero en la moneda correspondiente que ha pagado el usuario por el \textit{Viaje}.
            \item Usuarios. Datos necesarios para identificar a los viajantes (definido posteriormente).
            \item Asientos. Plazas asignadas a cada uno de los usuarios.
            \item Servicios adicionales. Servicios contratados por uno o varios compradores para utilización durante el
                el \textit{Viaje}.
        \end{itemize}
        Cada \textit{Billete} está asociado a un transporte.

    \item \textbf{Usuario.} Persona que utiliza la aplicación, o que es partícipe de algún \textit{Viaje} (puede ser
        añadida a uno por otro \textit{Usuario}). Contará con los siguientes atributos:
        \begin{itemize}
            \item Nombre. Una cadena de caracteres con el nombre completo de la persona.
            \item Documento de identidad. Una cadena de caracteres correspondiente al documento identificativo de la
                la región pertinente al \textit{Usuario}.
            \item Fecha de nacimiento. Formato usado en la región desde la que se utiliza la aplicación (en España por ejemplo,
                sería \textit{dd/mm/aaaa}).
            \item Correo electrónico. Método para contactar con el \textit{Usuario} en caso de ser necesario (por ejemplo, al
                enviar los billetes, o en caso de que se tenga que notificar la cancelación de un viaje). Será una cadena de
                caracteres que será confirmada a través de un correo electrónico.
            \item Reservas. Lista con instancias de la clase \textit{Reserva}, con los viajes que ha reservado el usuario.
        \end{itemize}
        Un usuario tiene uno o varios viajes asociados.
    
    \item \textbf{Notificación.} Mensaje recibido por un \textit{Usuario} acerca de su \textit{Viaje}. Sus atributos serán los siguientes.
        \begin{itemize}
            \item Mensaje. Una cadena de caracteres con el objeto de la notificación.
            \item Fecha. Fecha en la que se envió la notificación. Formato es el de la región en que se utiliza (en España por ejemplo,
                sería \textit{dd/mm/aaaa}).
            \item Hora. Hora a la que se envió la notificación. El formato será \textit{HH:MM}.
        \end{itemize}

        Una notificación va dirigida a un \textit{Usuario} y es acerca de un \textit{Viaje}.
    
    \item \textbf{Contacto con atención al cliente.} Utilizado en caso de que algún \textit{Usuario} tenga uno o varios problemas con la aplicación. Consta
        de los siguientes atributos:
        \begin{itemize}
            \item Mensajes. Cadenas de caracteres intercambiadas entre un \textit{Usuario} y las personas al cargo de la atención al cliente.
            \item Fecha. Fecha en la que se envió cada mensaje. Formato es el de la región en que se utiliza (en España por ejemplo,
                sería \textit{dd/mm/aaaa}).
            \item Hora. Hora a la que se envió cada mensaje. El formato será \textit{HH:MM}.
        \end{itemize}

        El contacto con atención al cliente lo hace un \textit{Usuario} con un problema con un \textit{Viaje}.

    \item \textbf{Lista de viajes.} Mostrada para que el \textit{Usuario} pueda elegir que viajes son los que desea. Tiene los siguientes atributos:
        \begin{itemize}
            \item Viaje de ida. Instancias del elemento \textit{Viaje} que tengan como origen la ciudad seleccionada por el \textit{Usuario}.
            \item Viaje de vuelta. Instancias del elemento \textit{Viaje} que tengan como destino la ciudad seleccionada por el \textit{Usuario}.
            \item Filtros. Estos representarán qué instancias de \textit{Viaje} se encuentran en la lista. Se podrá filtrar según precio (formato numérico
                de la cantidad de '€', o la moneda correspondiente), duración (formato \textit{HH:MM}), y si es accesible o no (una variable \textit{booleana}).
            \item Ordenación. Representa el orden en el que está ordenada la lista. Será un enumerado con cuatro opciones: \textit{Precio de mayor a menor}, 
                \textit{Precio de menor a mayor}, \textit{Duración de mayor a menor} y \textit{Duración de menor a mayor}.
        \end{itemize}

    La lista de viajes tiene asociados instancias de \textit{Viaje}.

    \item \textbf{Reserva.} Realizada por un \textit{Usuario} a partir de \textit{Viajes}. Consta de los siguientes atributos:
        \begin{itemize}
            \item Viajes. Instancias con los viajes que han sido reservados. En caso de haber varios acompañantes, cada uno tendrá su
                instancia de viaje.
            \item Usuarios. Instancias de \textit{Usuario} con los datos de los asistentes a cada uno de los viajes.
            \item Servicios. Servicios que hayan sido reservados por el \textit{Usuario} para el viaje.
        \end{itemize}

        La reserva relaciona los viajes con usuarios.

\end{itemize}
