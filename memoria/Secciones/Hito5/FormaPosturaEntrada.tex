\section{Factor de forma, postura y métodos de entrada}

Debido a que en el anterior hito estaban correctamente no hemos modificado nada, pero lo
incluimos en la memoria para mantener todos los apartados necesarios a la hora de hacer un prototipado.

Hemos creado los elementos de datos (transportes, billetes y viajes), junto con
sus atributos (por ejemplo, horario en el caso de transportes) y las relaciones
con otros elementos de datos, como los transportes y los viajes.

Posteriormente, se ha realizado la traducción de los requisitos funcionales en
elementos funcionales. De forma que describimos las acciones que podemos
realizar añadiendo una acción, un objeto y un contexto.

\subsection{Factor de forma}
Nuestra aplicación estará diseñada para móvil y ordenador. Si se utiliza desde
el ordenador, la aplicación se utilizará en mayoritariamente en casa para poder
organizar el viaje tranquilamente, aunque también puede utilizarse en un
contexto de trabajo en el que el usuario esté en una oficina. Si se utiliza la aplicación desde el
móvil el contexto cambia mucho, ya que normalmente el usuario hará un uso de la aplicación rápido para poder consultar información en cualquier parte o momento.

\subsection{Postura}
Por un lado tenemos varias \underline{posturas soberanas} centradas en toda la
parte de buscar destino, comparación de precios, ofertas o destinos a los que
acudir según las preferencias indicadas por el usuario. Por otro lado
tendríamos una \underline{postura temporal} relacionada con el chat de soporte
o con diferentes dudas que le puedan surgir al usuario que necesiten ser
subsanadas, así como diferentes reseñas que los usuarios hayan escrito sobre
los diferentes destinos. Por último tendríamos una \underline{postura demonio}
relacionada con las diferentes notificaciones que puedan surgir mediante el uso
de la aplicación (como la calificación o respuesta a una reseña hecha) y las
preferencias indicadas por el usuario.

\subsection{Métodos de entrada}
En caso de que se utilice la aplicación vía ordenador, los métodos de entrada
serán el teclado y ratón, mientras que si se accede a la aplicación por el
móvil, el método de entrada será la propia pantalla del teléfono.

\section{Elementos de datos y funcionales}

Los elementos de datos serán la fuente de información de la aplicación
(transportes, billetes, etc). Los elementos funcionales son las acciones que
pueden ejercer los usuarios sobre dichos objetos. Tras analizar lo visto en
apartados anteriores, estos son los distintos elementos de datos y los
elementos funcionales.