\section{Escenarios de validación}

Los escenarios de validación sirven para validar el diseño creado. Los escenarios de validación pueden incluir escenarios que representan variantes a la hora de realizar una acción, escenarios de acciones que es necesario realizar en el sistema pero que no forman parte de las necesidades del usuario, y escenarios de casos límite, que representan situaciones atípicas que el usuario puede llegar a encontrarse.

\textbf{Escenario 1:} ?`Qué pasaría si el usuario quiere buscar un viaje que sea directo? \\
\textbf{No contemplado.} De momento, el usuario no puede filtrar los transportes en función del número de paradas que tengan, tenemos que añadir filtro para buscar viajes directos.

\textbf{Escenario 2:} ?`Qué pasaría si el usuario se equivoca introduciendo sus datos personales de perfil? \\
\textbf{Existe.} Puede modificar sus datos desde la pestaña de usuario dándole al botón de modificar.

\textbf{Escenario 3:} ?`Qué pasaría si un usuario está en la página del comparador y entra en la pantalla de perfil y quiere volver? \\
\textbf{No contemplado.} De momento no contemplamos que se pueda volver a la pantalla anterior, tenemos que añadir botón de volver.

\textbf{Escenario 4:} ?`Qué pasaría si un usuario está en la página de reservas y quiere volver a su perfil? \\
\textbf{No contemplado.} De momento no contemplamos que se pueda volver a la pantalla anterior, tenemos que añadir botón de volver.

\textbf{Escenario 5:} ?`Qué pasaría si un usuario está en la página de una reserva concreta y quiere volver a la pantalla de todas sus reservas? \\
\textbf{No contemplado.} De momento no contemplamos que se pueda volver a la pantalla anterior, tenemos que añadir botón de volver.

\textbf{Escenario 6:} ?`Qué pasaría si el usuario quiere modificar sus datos porque ha visto un error en su billete? \\
\textbf{No contemplado.} No está diseñada la ventana pero el usuario podría desde su perfil modificar los datos de reserva y guardar sus cambios.

\textbf{Escenario 7:} ?`Qué pasaría si el usuario se da cuenta de que el precio final que aparece no se corresponde con el del billete por la aplicación de impuestos? \\
\textbf{No contemplado.} El precio que figure en los billetes debería aparecer con el IVA aplicado.

\textbf{Escenario 8:} ?`Qué pasaría si el usuario quiere consultar una duda concreta? \\
\textbf{Parcialmente contemplado.} Existe un botón de preguntas frecuentes pero no está implementado.

\textbf{Escenario 9:} ?`Qué pasaría si el usuario quiere consultar la fecha de sus viajes? \\
\textbf{No contemplado.} No está diseñada para ver las fechas en la sección de las reservas realizadas por el usuario.

\textbf{Escenario 10:} ?`Qué pasaría si el usuario quiere consultar las fechas en las que se produce el viaje cuando está buscando? \\
\textbf{No contemplado.} El usuario puede consultar una gran cantidad de datos en las tarjetas, pero las fechas del viaje no figuran.

\textbf{Escenario 11:} ?`Qué pasaría si el usuario quiere cambiar las fechas de una reserva? \\
\textbf{Contemplado.} No puede porque puede cambiar el precio, las horas, el transporte (por lo tanto, los asientos disponibles). Para ello habría que anular la reserva y coger otra.