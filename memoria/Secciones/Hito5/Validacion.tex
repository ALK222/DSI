\section{Escenarios de validación}

Los escenarios de validación sirven para validar el diseño creado. Los escenarios de validación
pueden incluir escenarios que representan variantes a la hora de realizar una acción, escenarios
de acciones que es necesario realizar en el sistema pero que no forman parte de las necesidades
del usuario, y escenarios de casos límite, que representan situaciones atípicas que el usuario
puede llegar a encontrarse.

\begin{escenario} % 11
    \centering
?`Qué pasaría si el usuario tiene un documento de identidad distinto al DNI?

\begin{solucion}
    \centering
    \textbf{Parcialmente contemplado.} Además del DNI, en \textit{Registro} está la opción de introducir tanto el NIE como el CIF. Otro tipo de documentos no estarían aceptados.
\end{solucion}
\end{escenario}

\begin{escenario} % 12
    \centering
?`Qué pasaría si el usuario quiere cambiar su contraseña?

\begin{solucion}
    \centering
    \textbf{Contemplado.} En la pantalla \textit{Modificar perfil}, el usuario tiene la oportunidad de cambiar la contraseña, además de otros datos del perfil.
\end{solucion}
\end{escenario}

\begin{escenario} % 13
    \centering
?`Qué pasaría si el usuario se confunde en el registro y selecciona que es discapacitado cuando no lo es?

\begin{solucion}
    \centering
    \textbf{No contemplado.} El usuario solo puede modificar sus datos personales.
\end{solucion}
\end{escenario}

\begin{escenario} % 14
    \centering
?`Qué pasaría si una de las personas que viaje no puede viajar?

\begin{solucion}
    \centering
    \textbf{No contemplado.} Solo se puede cancelar la reserva entera y no un viaje concreto.
\end{solucion}
\end{escenario}

\begin{escenario} % 15
    \centering
?`Qué pasa si un medio de transporte se retrasa o se cancela?

\begin{solucion}
    \centering
    \textbf{No contemplado.} Debería haber un apartado de información de qué medidas toma cada compañía con estos sucesos.
\end{solucion}
\end{escenario}

\begin{escenario} % 16
    \centering
?`Qué pasa si no se encuentran opciones para discapacitados?

\begin{solucion}
    \centering
    \textbf{No contemplado.} En caso de no encontrar ninguna opción de acuerdo a cierto tipo de
        filtrado se debería mostrar un mensaje de que no se ha encontrado ningún viaje, para no
        llevar al usuario a confusión de si ha habido algún fallo técnico al realizar el filtrado.
\end{solucion}
\end{escenario}

\begin{escenario} % 17
    \centering
?`Qué pasa si el usuario quiere cerrar sesión?

\begin{solucion}
    \centering
    \textbf{Contemplado.} Hay una opción de cerrar sesión en \textit{Perfil}.
\end{solucion}
\end{escenario}

\begin{escenario} % 18
    \centering
?`Qué pasa si se corta la conexión en medio de una reserva?

\begin{solucion}
    \centering
    \textbf{No contemplado.} Deberíamos poner una ventana que muestre este error de conexión explicándole al usuario lo que debería hacer.
\end{solucion}
\end{escenario}