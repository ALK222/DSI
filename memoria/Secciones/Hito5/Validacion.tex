\section{Escenarios de validación}

Los escenarios de validación sirven para validar el diseño creado. Los escenarios de validación pueden incluir escenarios que representan variantes a la hora de realizar una acción, escenarios de acciones que es necesario realizar en el sistema pero que no forman parte de las necesidades del usuario, y escenarios de casos límite, que representan situaciones atípicas que el usuario puede llegar a encontrarse.

En nuestro caso, nos reunimos por google meet para poner en común diferentes casos y hemos descrito una serie de escenarios en los que se hacen preguntas que no hemos contemplado en los escenarios de contexto.

Para comenzar vamos a coger los escenarios que hemos corregido de la anterior fase.

\begin{escenario} % 1
    \centering
    ?`Qué pasaría si el usuario quiere buscar un viaje que sea directo?

    \begin{solucion}
        \centering
        \textbf{Contemplado.} El usuario puede filtrar desde la pantalla del comparador la opción de solo viajes directos.
    \end{solucion}
\end{escenario}


\begin{escenario}
    % 2
    \centering
    ?`Qué pasaría si un usuario está en la página del comparador y entra en la pantalla de perfil y quiere volver?

    \begin{solucion}
        \centering
        \textbf{Contemplado.} El usuario desde la página de perfil tiene un botón para volver atrás a la pestaña en la que estaba, en este caso la del comparador.
    \end{solucion}
\end{escenario}

\begin{escenario}
    % 3
    \centering
    ?`Qué pasaría si un usuario está en la página de reservas y quiere volver a su perfil?

    \begin{solucion}
        \centering
        \textbf{Contemplado.} El usuario desde la página de reservas tiene un botón para volver atrás a la pestaña en la que estaba, en este caso la del perfil.
    \end{solucion}
\end{escenario}

\begin{escenario}
    % 4
    \centering
    ?`Qué pasaría si un usuario está en la página de una reserva concreta y quiere volver a la pantalla de todas sus reservas?

    \begin{solucion}
        \centering
        \textbf{Contemplado.} El usuario desde la reserva concreta puede darle a la cruz para ver todas sus reservas otra vez.
    \end{solucion}
\end{escenario}

\begin{escenario}
    % 5
    \centering
    ?`Qué pasaría si el usuario quiere modificar sus datos porque ha visto un error en su billete?

    \begin{solucion}
        \centering
        \textbf{Contemplado.} No está diseñada la ventana pero el usuario podría desde su perfil modificar los datos de reserva y guardar sus cambios.
    \end{solucion}
\end{escenario}

\begin{escenario}
    % 6
    \centering
    ?`Qué pasaría si el usuario quiere modificar sus datos porque ha visto un error en su billete?

    \begin{solucion}
        \centering
        \textbf{Contemplado.} El usuario desde sus reservas puede modificar el viaje, cambiando los datos de los pasajeros, el asiento y los servicios adicionales.
    \end{solucion}
\end{escenario}

\begin{escenario}
    % 7
    \centering
    ?`Qué pasaría si el usuario se da cuenta de que el precio final que aparece no se corresponde con el del billete por la aplicación de impuestos?

    \begin{solucion}
        \centering
        \textbf{Contemplado.} El precio que figure en los billetes aparece con el IVA aplicado.
    \end{solucion}
\end{escenario}

\begin{escenario}
    % 8
    \centering
    ?`Qué pasaría si el usuario quiere consultar una duda concreta?

    \begin{solucion}
        \centering
        \textbf{Contemplado.} Existe un botón de preguntas frecuentes. Si la duda del usuario no estuviera incluida, hay un servicio de atención al cliente por el que, tanto a través de llamada como del chat, puede preguntar.
    \end{solucion}
\end{escenario}

\begin{escenario}
    % 9
    \centering
    ?`Qué pasaría si el usuario quiere consultar la fecha de sus viajes?

    \begin{solucion}
        \centering
        \textbf{Contemplado.} Desde la página de mis reservas se pueden ver las fechas concretas de los viajes.
    \end{solucion}
\end{escenario}

\begin{escenario}
    % 10
    \centering
    ?`Qué pasaría si el usuario quiere consultar las fechas en las que se produce el viaje cuando está buscando?

    \begin{solucion}
        \centering
        \textbf{Contemplado.} El usuario puede consultar una gran cantidad de datos en las tarjetas, las fechas del viaje figuran, como las horas, el precio, la compañía.
    \end{solucion}
\end{escenario}

\begin{escenario} % 11
    \centering
    ?`Qué pasaría si el usuario tiene un documento de identidad distinto al DNI?

    \begin{solucion}
        \centering
        \textbf{Parcialmente contemplado.} Además del DNI, en \textit{Registro} está la opción de introducir tanto el NIE como el CIF. Otro tipo de documentos no estarían aceptados.
    \end{solucion}
\end{escenario}

\begin{escenario} % 12
    \centering
    ?`Qué pasaría si el usuario quiere cambiar su contraseña?

    \begin{solucion}
        \centering
        \textbf{Contemplado.} En la pantalla \textit{Modificar perfil}, el usuario tiene la oportunidad de cambiar la contraseña, además de otros datos del perfil.
    \end{solucion}
\end{escenario}

\begin{escenario} % 13
    \centering
    ?`Qué pasaría si el usuario se confunde en el registro y selecciona que es discapacitado cuando no lo es?

    \begin{solucion}
        \centering
        \textbf{No contemplado.} El usuario solo puede modificar sus datos personales.
    \end{solucion}
\end{escenario}

\begin{escenario} % 14
    \centering
    ?`Qué pasaría si una de las personas que viaje no puede viajar?

    \begin{solucion}
        \centering
        \textbf{No contemplado.} Solo se puede cancelar la reserva entera y no un viaje concreto.
    \end{solucion}
\end{escenario}

\begin{escenario} % 15
    \centering
    ?`Qué pasa si un medio de transporte se retrasa o se cancela?

    \begin{solucion}
        \centering
        \textbf{No contemplado.} Debería haber un apartado de información de qué medidas toma cada compañía con estos sucesos.
    \end{solucion}
\end{escenario}

\begin{escenario} % 16
    \centering
    ?`Qué pasa si no se encuentran opciones para personas con discapacidad?

    \begin{solucion}
        \centering
        \textbf{No contemplado.} En caso de no encontrar ninguna opción de acuerdo a cierto tipo de
        filtrado se debería mostrar un mensaje de que no se ha encontrado ningún viaje, para no
        llevar al usuario a confusión de si ha habido algún fallo técnico al realizar el filtrado.
    \end{solucion}
\end{escenario}

\begin{escenario} % 17
    \centering
    ?`Qué pasa si el usuario quiere cerrar sesión?

    \begin{solucion}
        \centering
        \textbf{Contemplado.} Hay una opción de cerrar sesión en \textit{Perfil}.
    \end{solucion}
\end{escenario}

\begin{escenario} % 18
    \centering
    ?`Qué pasa si se corta la conexión en medio de una reserva?

    \begin{solucion}
        \centering
        \textbf{No contemplado.} Deberíamos poner una ventana que muestre este error de conexión explicándole al usuario lo que debería hacer.
    \end{solucion}
\end{escenario}

Tras haber puesto en común los escenarios de validación que hemos podido identificar entre todos, hemos podido notificar que la parte de gestión 
de situaciones imprevistas dentro de nuestra aplicación no está correctamente gestionada, lo que puede dar lugar a problemas con los distintos usuarios,
que pueden desistir de usar nuestra aplicación en el caso de que les ocurra alguno de estos imprevistos. Es por ello que hemos considerado necesario realizar
una nueva iteración en la que se pongan solución a estos problemas.