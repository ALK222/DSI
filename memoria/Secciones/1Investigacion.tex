\chapterA{Investigación}

\section{Introducción}

Para poder diseñar una aplicación correctamente, es muy importante realizar previamente una investigación para saber qué
es lo que realmente se necesita, y cuáles son los problemas de nuestro público objetivo. Hay muchas maneras de conseguir esto, pero en nuestro
caso, como desgraciadamente no disponemos del tiempo para poder usar todos los métodos, hemos realizado las siguientes.

\begin{itemize}
    \item \textbf{Entrevistas.} Es una de las partes más importantes de la investigación. Con estas entrevistas podremos sacar distintos tipos de usuarios y distintos casos de uso de los mismos. Aspiramos a tener 6 entrevistas.
    \item \textbf{Cuestionarios.} Con estos cuestionarios podemos conseguir una información más cerrada. Esta información puede ser interesante para comparar nuestra idea contra otras aplicaciones existentes en el mercado.
\end{itemize}



Pero antes de realizar esta labor, debemos saber reconocer cuál es el público objetivo de nuestra aplicación y
de qué manera podemos clasificar a los distintos perfiles dentro de los clientes potenciales. Para eso hemos realizado la
\textbf{Hipótesis de personas}.

\section{Hipótesis de personas}

En esta primera fase de investigación, el primer paso que vamos a seguir es la identificación de los posibles usuarios que vamos a tener
en nuestra aplicación. Nuestro principal objetivo, como hemos visto anteriormente, es ofrecer una herramienta que permita a las personas con
discapacidad intelectual tener la posibilidad de utilizar un comparador de viajes sin problemas. Los principales usuarios que hemos identificado
y los cuáles vamos a entrevistar en la posterior fase de entrevistas son los siguientes:

\begin{itemize}
    \item\textbf{Gente que viaja por negocio.} Gente que tiene que planear viajes por motivos laborales, bien sea para viajes nacionales o internacionales. Normalmente serán viajes individuales.
    \item\textbf{Gente que viaja por ocio.} Personas que viajan por turismo, normalmente en grupo.
    \item\textbf{Gente que viaja para ver a sus familiares.} Viajero recurrente.
\end{itemize}

Por otro lado, vamos a tener otros tipos de personas identificados, que no van a ser usuarios potenciales de nuestra aplicación pese a pertenecer a 
alguno de los anteriores grupos, por lo que en el momento que detectemos que se trata de una persona encuadrada en uno de estos tipos
vamos a finalizar la entrevista ó el cuestionario, ya que no vamos a poder extraer información de valor para nuestra aplicación.


\begin{itemize}
    \item {\textbf{Usuarios que prefieren viajar con todo planificado por una agencia.}} Se trata de aquellos usuarios que cada vez que quieren
        reservar un viaje no les importa realizar un gasto extra y prefieren que todo sea organizado por una agencia de viajes, sobre todo de cara a
        evitar la aparición de ciertos conflictos que pueden tener con otras aplicaciones de la competencia.
    \item {\textbf{Usuarios que no les guste viajar y no tengan la necesidad.}} Existen usuarios que no les gusta viajar y que además nunca se han visto
        (ni se van a ver en un futuro próximo), por lo que no nos van a resultar de interés para la aplicación, ya que el objetivo buscado son perfiles que
        hayan experimentado el proceso y puedan contarnos aquellos inconvenientes que han podido encontrarse a lo largo del proceso.
\end{itemize}
 
A modo de conclusión, los perfiles de usuario que tenemos, como se puede apreciar está influenciado por dos factores muy importantes: la necesidad (o ausencia de ella) para viajar; gente que viaja por motivos laborales o familiares y gente que viaja por placer y ocio, y

\section{Plan de investigación}

Para la investigación de \textit{\app} usaremos dos técnicas: entrevistas y cuestionarios. En las subsecciones \ref{subsec:entrevistas} y \ref{subsec:cuestionarios} hablaremos en detalle del número de instancias de cada técnica y de sus distintos objetivos en más detalle.

\subsection{Entrevistas} \label{subsec:entrevistas}

Las entrevistas serán el método principal de obtención de datos. Con las entrevistas podremos ver usuarios potenciales (y no potenciales) para la aplicación y podremos hacer preguntas con mayor detalle. Se fija el objetivo en 6 entrevistas.

Las tareas a realizar en las entrevistas son:
\begin{itemize}
    \item Crear un guion de entrevista con diversas preguntas para los distintos tipos de usuarios.
    \item Reclutar usuarios de distintos clases para poder tener una imagen de cada tipo de usuario de la aplicación.
    \item Resumir las entrevistas y sacar los datos más importantes de las mismas.
    \item Realizar el mapa de empatía de cada usuario entrevistado.
\end{itemize}

FALTA ENCARGADOS DE LA TAREA, CUANDO SE VAN A REALIZAR Y TIEMPO QUE SE VA A DEDICAR

\subsection{Cuestionarios} \label{subsec:cuestionarios}

\textbf{TODO}

%%%%%%%%%%%%%%%%%%%%%%%%%%%%%%%%%%%%%%%%%%%%%%%%%%%%%%%%%%%%%%
\section{Entrevistas}
%%%%%%%%%%%%%%%%%%%%%%%%%%%%%%%%%%%%%%%%%%%%%%%%%%%%%%%%%%%%%%

Es la parte más importante de la investigación, ya que es de donde conseguiremos obtener más información. Consiste en realizar una serie de preguntas al usuario para
ver si encaja con los perfiles objetivo de nuestra aplicación, y en caso de hacerlo, obtener los datos necesarios para poder diseñarla. Gracias a esto podremos averiguar
cuáles son los problemas que tienen estas personas con los comparadores actuales y qué necesidades tendrían.

Tenemos distintos clientes que forman parte de nuestra hipótesis de personas, y ambos tienen necesidades distintas. Por tanto tendremos distintas preguntas
dependiendo del perfil al que nos enfrentemos.

Todas las entrevistas comenzarán presentándonos y preguntando el nombre al entrevistado. Tras eso, tendremos que pedir autorización para grabar imágenes, ya que las
grabaciones son necesarias para un posterior análisis y recabar así la mayor cantidad de información posible. Para que la entrevista sea más distendida, preguntaremos
si le podemos tutear. Tras esto, explicaremos nuestros objetivos con la aplicación y procederemos a realizar las preguntas.

\subsection{Preguntas a usuarios}

Al comienzo realizaremos el denominado como \textit{Screener}, es decir, una serie de preguntas para ver si el entrevistado es un cliente potencial. En caso de serlo,
podremos proceder con el resto de preguntas.

\begin{enumerate}
    \item {\textbf{?`Cuántos años tienes?}} sirve para encuadrar al usuario dentro de un marco de edad concreto y poder tratar con él en función de esto.
    \item {\textbf{?`Cómo de cómodo te sientes con la tecnología?}}
    \item {\textbf{?`Te gusta viajar? ?`Cuéntame por qué?}} es la pregunta que nos va a determinar si el usuario es potencial de la aplicación
                o si bien lo tenemos que descartar.
    \begin{enumerate}
        \item {\textit{Sí,}} usuario potencial. Tenemos que conocer ahora si le gusta viajar por ocio o bien lo tiene que hacer por negocios.
        \item {\textit{No,}} puede seguir estando dentro de la hipótesis de usuarios. En este caso las preguntas a hacer van a variar y van a
                        depender de la respuesta que nos de.
    \end{enumerate}
    \item {\textbf{?`Suele viajar?}} sirve para identificar si el usuario es apto, porque si no viaja no tiene sentido la aplicación.
    \item {\textbf{?`Le gustaría viajar más?}} sirve para saber si el usuario que no viaja tiene pensado viajar en un futuro y por tanto, considerarlo apto.
    \item {\textbf{?`Disfruta cuando viaja?}} sirve para entender al usuario que viaja y si va a usar la aplicación más o menos frecuente.
    \item {\textbf{?`Cuál es tu medio de transporte favorito y por qué?}} sirve para entender al usuario que viaja y si va a usar la aplicación más o menos frecuente.
    \item {\textbf{?`Qué tipo de viajes has hecho?}} sirve para entender al usuario que viaja y si va a usar la aplicación más o menos frecuente.
    \item {\textbf{?`Sueles viajar acompañado?}} tenemos que tener en cuenta si la persona con la que estamos tratando requiere de la
                ayuda de un acompañante que viaje con él para tenerlo en cuenta a la hora de desarrollar la aplicación y nos ayuda a conocer un poco al entrevistado.
    \item {\textbf{?`Por qué motivos suele viajar?}} sirve para identificar las motivaciones del usuario.
    \item {\textbf{?`Qué es lo que más le dificulta a la hora de viajar? ?`Hay algo más que le dificulte viajar?}} sirve para identificar molestias que tiene
                el usuario al planificar un viaje.
    \item {\textbf{?`Cuando vas a organizar un viaje, que es lo primero que haces?}}
    \item {\textbf{?`Te encargas tú de organizar el viaje?}} queremos conocer si el usuario tiene la iniciativa para organizar el viaje por sí solo o bien si
                recurre a profesionales como agencias de viajes o a terceras personas.
    \item{ \textbf{?`Cómo has organizado tus viajes?}}
    \begin{enumerate}
        \item {\textit{No:}} ?`No has pensado nunca en usar un comparador de viajes? queremos conocer si aunque el usuario recurra a agencias de viajes o
                        a otras personas para realizar el viaje usaría en algún momento nuestra aplicación.
        \item {\textit{Si:}} ?`Te has encontrado alguna dificultad en el proceso? está bien para finalizar el screener e introducir la siguiente parte.
    \end{enumerate}
    \item {\textbf{?`Te resulta más cómodo realizar estas búsquedas de viajes en una aplicación móvil o en una página web?}}
    \item {\textbf{?`Cuál es el factor clave que hace que se decante por esa opción en un viaje?}} sirve para saber sus prioridades.
    \item {\textbf{?`Influye el coste del viaje en su elección?}} sirve para saber más información.
    \item {\textbf{?`Cuál de las partes de una página tradicional de comparación de viajes te parece más tediosa?}} necesitamos conocer los problemas que
                puede encontrarse el usuario en las páginas tradicionales para tenerlo en cuenta y poder mejorarlo en nuestra aplicación. En caso de que
                la respuesta sea afirmativa, podemos preguntarle si existe alguna opción de ayuda dentro de la plataforma.
    \item {\textbf{En caso de que hayas tenido algún problema en estas páginas, ?`has podido solicitar ayuda de manera sencilla?}}
    \item {\textbf{?`Según tu opinión, ?`cómo debería ser la forma ideal en la que una aplicación muestre la información?}}
    \item {\textbf{?`Hay alguna función de las páginas tradicionales que consideres útil para tus necesidades?}}
    \item {\textbf{En caso de que hayas realizado alguna reserva de viaje, ?`has conocido y sido informado de forma clara de las condiciones
                        y políticas de cancelación del viaje?}}
    \item {\textbf{?`Te gustaría que estas páginas incluyesen más información sobre accesibilidad para viajeros con discapacidad?}}
    \item {\textbf{?`Podrías darme algún ejemplo de aplicación que te guste y uses a diario?}}: queremos poner al usuario en una situación
    en la que nos comente una aplicación que le guste para poder conocer los motivos que le llevan a ello.
    \item {\textbf{?`Consideras que es una aplicación accesible??`Por qué?}}: queremos conocer desde el punto de vista de la persona aquellos
    elementos y problemas que puede identificar dentro de la aplicación y que puedan suponer un problema.
    \item {\textbf{Acordarse de hacer el debriefing, para hacer un repaso de lo que ha dicho a ver si se acuerda de algo más.}}
    \item {\textbf{?`Se te ocurre algo más de lo que hemos hablado que podría ayudarnos?}}
\end{enumerate}

Al finalizar, le agradeceremos al entrevistado su tiempo y su participación en nuestro proyecto.


\subsection{Resumenes de entrevistas}

\section{Cuestionarios}

Los cuestionarios los hemos realizado finalmente con Formularios de \textit{Google}. El público objetivo de este cuestionario es gente de cualquier edad a la que le guste viajar. Este cuestionario se ha difundido por grupos de amigos, grupos de asociaciones de la facultad y a distintos familiares por \textit{Whatsapp}.

El público que buscamos para realizar este cuestionario eran personas que viajen al menos una vez al año y utilicen aplicaciones de comparación de algún tipo. Al ser un espectro bastante amplio podemos permitirnos difundir e formulario por un medio bastante amplio.

La finalidad de este cuestionario es el sacar información más concreta (frecuencia de viajes, aplicaciones usadas para estos) de un gran número de usuarios. El ``gran número'' de usuarios ha sido finalmente de 59 respuestas, de las cuales 6 son viajeros por trabajo y 52 viajantes por ocio, solo uno ha seleccionado la opción de ``otro''.

\subsection{Guión}
Las preguntas del cuestionario son las siguientes:
\begin{enumerate}
    \item\textbf{Edad.} Pregunta de screening para enmarcar el resto de la respuesta.
    \item\textbf{Género.} Una vez más, pregunta de screening.
    \item\textbf{Entorno en el que vive el encuestado.} Otra pregunta de screening para saber si el usuario vive en una ciudad o en el entorno rural.
    \item\textbf{Poder adquisitivo.} No entramos en muchos detalles, pero no es lo mismo la comparativa que puede hacer una persona con unos recursos bajos y otra con unos recursos mayores. (observación). Pregunta de screening.
    \item\textbf{Gusto por viajar.} Es importante el contexto de si te gusta o no viajar para poder entender tus frustraciones con los sistemas de comparación.
    \item\textbf{La parte buena (o la parte mala) de viajar.} Queremos conocer las partes que más frustran a los viajantes para ver si es algo que nuestra aplicación pueda solucionar.
    \item\textbf{Frecuencia de viaje.} La gente que más viaje será la que más compare (intuitivamente), así que es importante saber la frecuencia en la que viaja un encuestado para tener más perspectiva sobre su opinión.
    \item\textbf{Posibilidad de viajar más.} Queremos saber si nuestros usuarios quieren viajar más y hay algo que se lo impida.
    \item\textbf{Disfrute al viajar.} Parece similar a la  pregunta 5, pero a una persona puede no gustarle alguna parte del proceso del viaje y aun así disfrutar del resto del viaje.
    \item\textbf{Medios de transporte.} Al ser nuestra aplicación un comparador de viajes, necesitamos saber cuales son los medios de transporte más usados.
    \item\textbf{Motivos de viaje.} El viaje por ocio y el viaje por trabajo pueden presentar maneras muy distintas de como enfocar esa comparación y la forma en la que buscamos destino y medio de transporte.
    \item\textbf{Uso de herramientas y dificultad de las mismas.} Para hacer  una buena aplicación hay que conocer a la competencia, por ello queremos saber la opinión de nuestro público sobre otras aplicaciones populares.
    \item\textbf{Preguntas para gente con discapacidad:}
    \begin{enumerate}
        \item\textbf{Discapacidad, tipo (general) de discapacidad y adaptaciones.} Es importante para nosotros el saber que problemas pueden surgirle a alguien con algún tipo de discapacidad para tenerlo en cuenta en la aplicación.
    \item\textbf{Frecuencia en la que el encuestado organiza viajes.} Si el encuestado no organiza viajes es complicado que nos pueda dar mucho \textit{feedback} sobre aplicaciones de comparativa de viajes.
    \item\textbf{Dificultad en a la hora de buscar viajes y el tipo de dificultad.} No van a ser las mismas dificultades para gente con discapacidades físicas que para gente con discapacidades psíquicas.
    \item\textbf{Falta de funcionalidades en los buscadores tradicionales.} Queremos saber si los usuarios han echado en falta alguna característica en sus búsquedas.
    \end{enumerate}
    \item\textbf{Preguntas para gente sin discapacidades}
    \begin{enumerate}
        \item\textbf{Viajes en el último año.} Si un usuario ha viajado en el último año nos puede dar información bastante actualizada.
        \item\textbf{Organización del viaje.} Queremos saber si el usuario ha organizado algún viaje en el último año.
        \item\textbf{Uso de comparadores, agencias o ninguna de las dos opciones y sus principales motivos para ello.} Queremos saber si los usuarios usan comparadores y sus principales motivos para ello.
        \item\textbf{Problemas y accesibilidad de los comparadores.} Queremos saber la dificultad de los usuarios de la encuesta ante distintas características de los comparadores modernos.
        \item\textbf{Falta de funcionalidades.} Queremos saber si los usuarios han echado de menos alguna funcionalidad concreta en los comparadores tradicionales.
    \end{enumerate}
    
\end{enumerate}

Podemos ver que la mayor parte de la población viaja por ocio y se considera de clase media. Vemos también que la gente está bastante contenta con los comparadores actuales a los que solo le harían cambios menores o simplificaciones. La mayoría de usuarios viajan en coche propio o en transporte público (tren o bus). También vemos un uso prominente de comparadores de viajes para ahorrar tiempo y dinero.

\subsection{Enlaces}

Dejamos aquí los enlaces al cuestionario en bruto \footnote{Enlace al cuestionario: \url{https://drive.google.com/drive/folders/1btqEATkoGqjZbF4wd30f3foDVuh37EUI?usp=drive_link}} y a los resultados en una hoja de calculo \footnote{ Enlace a los resultados en bruto: \url{https://docs.google.com/spreadsheets/d/1pTHiN6l9IVTqoC7u9HCUusoRIATDoZgKK113m_7aHrQ/edit?usp=drive_link}}

\section{Lista de factoides}

\section{Analisis de la competencia}

Para identificar a la competencia primero debemos conocer cuales son las funcionalidades que va a tener nuestra aplicación para ver cuáles otras del mismo tipo se parecen. Las funcionalidades principales de nuestra aplicación serían:
\begin{itemize}
    \item Búsqueda de alojamiento.
    \item Búsqueda de medio de transporte.
    \item Comparar precios para el mismo viaje.
    \item Comprar billetes y alojamientos.
\end{itemize}

Hemos hecho una búsqueda intensiva de las aplicaciones que pueden tener similitudes con nuestra aplicación.

Vamos a diferenciar entre competencia total y competencia parcial. Las aplicaciones que definimos con competencia total, son aquellas que comparten casi en su totalidad las mismas funcionalidades que nuestra aplicación. Por otro lado, hemos definido como competencia parcial aquellas que comparten alguna funcionalidad similar a nuestra aplicación (solo búsqueda de medios de transporte o solo búsqueda de alojamiento).

\subsection{Competencia total}
\begin{itemize}
    \item\textbf{Kayak.} Es una aplicación que permite comparar alojamientos y medios de transporte, ofrece descripciones detalladas de los destinos con cosas que se pueden hacer en los lugares y cómo moverse por allí. Tiene un mapamundi con etiquetas de los sitios con los precios más baratos. Tiene un 1,8 de valoración sobre 5.
    \item\textbf{eDreams.} Es un comparador de viajes que ofrece búsqueda de transporte y alojamiento y además ofrece el paquete conjunto de las dos y también permite alquiler de coche. Tiene servicio prime para reducir el coste de los viajes.Tiene un 4 de valoración sobre 5.
    \item\textbf{Momondo.} Es un comparador de vuelos, alojamiento y ofrece viajes completos. Tiene una opción de alertas de precios. Tiene un 4 sobre 5 de valoración.
    \item\textbf{SkyScanner.} Es una aplicación para buscar transporte y alojamiento en un destino determinado en unas fechas determinadas ofrece las últimas ofertas y tiene una sección de recomendaciones. Tiene un 3,9 sobre 5 de valoración.
\end{itemize}

\subsection{Competencia parcial:}
\begin{itemize}
    \item\textbf{Trivago.} Es una aplicación para buscar y comparar alojamientos, tiene un sistema de opiniones para valorar el alojamiento, ofrece ofertas, tiene diferentes opciones para encontrar un alojamiento adecuado para las necesidades del usuario. Tiene un 2,8 de valoración sobre 5.
    \item\textbf{Iberia.} Es una aplicación de la aerolínea que permite comparar vuelos de Iberia o de sus filiales. también tiene Iberia plus con un sistema de puntos para obtener descuentos. Tiene un 3 de valoración sobre 5
    \item\textbf{Booking.} Esta web permite comparar y comprar sólo alojamiento, tiene colaboración con aerolíneas como Iberia para hacer descuentos. Tiene un 1,2 de valoración sobre 5.
\end{itemize}

\subsection{Dimensiones}

\textbf{Comparar transporte.} Realizar una comparación de múltiples medios de transporte en unas fechas. [Respuesta binaria, tabla \ref{table:comp-transporte}]
\begin{itemize}
    \item Vuelos
    \item Trenes
    \item Buses
    \item Flexibilidad de fechas 
    \item Filtros de búsqueda 
    \item Alquiler de coche 
\end{itemize}

\begin{table}[H]
    \centering
    \begin{tabular}{l|l|l|l|l|l|l|}
    \cline{2-7}
                                     & Vuelos & Trenes & Buses & Flexibilidad de fechas & Filtros de búsqueda & Alquiler de coche \\ \hline
    \multicolumn{1}{|l|}{Kayak}      & Si     & Si     & Si    & Si                     & Si                  & Si                \\ \hline
    \multicolumn{1}{|l|}{eDreams}    & Si     & Si     & No    & No                     & Si                  & Si                \\ \hline
    \multicolumn{1}{|l|}{Momondo}    & Si     & Si     & Si    & Si                     & Si                  & Si                \\ \hline
    \multicolumn{1}{|l|}{SkyScanner} & Si     & No     & No    & No                     & Si                  & Si                \\ \hline
    \multicolumn{1}{|l|}{Trivago}    & No     & No     & No    & No                     & No                  & No                \\ \hline
    \multicolumn{1}{|l|}{Iberia}     & Si     & No     & No    & Si                     & Si                  & No                \\ \hline
    \multicolumn{1}{|l|}{Booking}    & No     & No     & No    & No                     & No                  & No                \\ \hline
    \end{tabular}
    \caption{Tabla de posibilidad de comparación de transportes entre distintas plataformas}
    \label{table:comp-transporte}
\end{table}

\textbf{Comprar transporte.} Poder comprar un medio de transporte previamente comparado. [Respuesta binaria, tabla \ref{table:comprar-transporte}]

\begin{itemize}
    \item Sistema de puntos
    \item Suscripción prime 
    \item Ecologismo 
    \item Seguros
\end{itemize}

\begin{table}[H]
    \centering
    \begin{tabular}{l|l|l|l|l|}
    \cline{2-5}
                                     & Sistemas de puntos & Suscripción prime & Ecologismo & Seguros \\ \hline
    \multicolumn{1}{|l|}{Kayak}      & No                 & No                & No         & Si      \\ \hline
    \multicolumn{1}{|l|}{eDreams}    & Si                 & Si                & Si         & Si      \\ \hline
    \multicolumn{1}{|l|}{Momondo}    & No                 & No                & No         & Si      \\ \hline
    \multicolumn{1}{|l|}{SkyScanner} & Si                 & Si                & Si         & Si      \\ \hline
    \multicolumn{1}{|l|}{Trivago}    & No                 & No                & No         & No      \\ \hline
    \multicolumn{1}{|l|}{Iberia}     & Si                 & Si                & No         & Si      \\ \hline
    \multicolumn{1}{|l|}{Booking}    & No                 & No                & No         & No      \\ \hline
    \end{tabular}
    \caption{Tabla comparativa sobre distintas características de los sistemas de compra}
    \label{table:comprar-transporte}
    \end{table}

\textbf{Comparar alojamiento.} Realizar una comparación dde múltiples alojamientos en unas fechas. [Respuesta binaria, tabla \ref{table:comp-aloj}]

\begin{itemize}
    \item Opiniones 
    \item Valoración con estrellas 
    \item Filtros de búsqueda 
    \item Flexibilidad de fechas
\end{itemize}

\begin{table}[H]
    \centering
    \begin{tabular}{l|l|l|l|l|}
    \cline{2-5}
                                     & Opiniones & Valoración con estrellas & Filtros de búsqueda & Flexibilidad de fechas \\ \hline
    \multicolumn{1}{|l|}{Kayak}      & Si        & Si                       & Si                  & Si                     \\ \hline
    \multicolumn{1}{|l|}{eDreams}    & Si        & Si                       & Si                  & No                     \\ \hline
    \multicolumn{1}{|l|}{Momondo}    & Si        & Si                       & Si                  & Si                     \\ \hline
    \multicolumn{1}{|l|}{SkyScanner} & Si        & Si                       & Si                  & No                     \\ \hline
    \multicolumn{1}{|l|}{Trivago}    & Si        & Si                       & Si                  & No                     \\ \hline
    \multicolumn{1}{|l|}{Iberia}     & No        & No                       & No                  & No                     \\ \hline
    \multicolumn{1}{|l|}{Booking}    & Si        & Si                       & Si                  & Si                     \\ \hline
    \end{tabular}
    \caption{Tabla de funcionalidades de la comparación de viajes}
    \label{table:comp-aloj}
    \end{table}

\textbf{Comprar alojamiento.} Poder comprar un alojamiento previamente comparado. [Respuesta vinaria, tabla \ref{table:comprar-aloj}]

\begin{itemize}
    \item Sistema de puntos 
    \item Suscripción prime 
    \item Seguros
\end{itemize}

\begin{table}[H]
    \centering
    \begin{tabular}{l|l|l|l|}
    \cline{2-4}
                                     & Sistemas de puntos & Suscripción prime & Seguros \\ \hline
    \multicolumn{1}{|l|}{Kayak}      & No                 & No                & Si      \\ \hline
    \multicolumn{1}{|l|}{eDreams}    & No                 & Si                & Si      \\ \hline
    \multicolumn{1}{|l|}{Momondo}    & No                 & No                & Si      \\ \hline
    \multicolumn{1}{|l|}{SkyScanner} & Si                 & Si                & Si      \\ \hline
    \multicolumn{1}{|l|}{Trivago}    & No                 & No                & No      \\ \hline
    \multicolumn{1}{|l|}{Iberia}     & No                 & No                & No      \\ \hline
    \multicolumn{1}{|l|}{Booking}    & No                 & No                & No      \\ \hline
    \end{tabular}
    \caption{Tabla de comparación de las distintas funcionalidades de la compra / reserva de alojamientos}
    \label{table:comprar-aloj}
    \end{table}

\subsection{Recomendaciones de acción}

A raíz de los resultados obtenidos en las tablas vamos a sacar unas recomendaciones de acción que nos permitan saber qué nos quiere decir este estudio.

\begin{itemize}
    \item Todas las páginas que ofrecen servicio de comparar transporte tienen buenos filtros de búsqueda, debemos incluir un buen sistema de filtros.Todas ofrecen como mínimo poder buscar vuelos, algunas incluyen trenes y buses en las búsquedas, sería interesante incluir todas en nuestra aplicación. No todas ofrecen poder buscar con flexibilidad de fechas, así que vemos interesante incluir flexibilidad de fechas en nuestra aplicación. No todas ofrecen servicio de alquiler de coche, es interesante incluirlo en la aplicación.
    \item Muy pocas páginas de las estudiadas ofrecen un sistema de puntos y suscripción prime, puede ser interesante incluirlo en nuestra aplicación. Solo dos de las páginas estudiadas ofrecen opciones de ecologismo, valoramos que no es tan interesante incluirlo en nuestra aplicación. Casi todas ofrecen un sistema de seguros a la hora de comprar un medio de transporte, es interesante incluirlo.
    \item Todas las aplicaciones utilizan un sistema de opiniones y de valoración gráfica con estrellas a la hora de mostrar los alojamientos comparados. Es muy interesante incluir esto en nuestra aplicación. Así como filtros de búsqueda también tienen todas, es interesante incluir buenos filtros. No todas ofrecen flexibilidad de fechas a la hora de comparar los alojamientos, es interesante incluirlo en nuestra aplicación.
    \item Para comprar alojamientos solo una aplicación incluye un sistema de puntos para comparar alojamientos, puede no ser interesante incluir un sistema de puntos para los alojamientos. De las páginas estudiadas para comprar alojamiento, dos de ellas utilizan sistema de prime, es interesante incluirlo. De estas páginas para comprar alojamiento todas tienen un sistema de seguros, es muy interesante incluirlo.
\end{itemize}

\section{Mapas de empatía}
