\chapterA{Investigación}

\section{Introducción}

Para poder diseñar una aplicación correctamente, es muy importante realizar previamente una investigación para saber qué
es lo que realmente se necesita, y cuáles son los problemas de nuestro público objetivo. Hay muchas maneras de conseguir esto, pero en nuestro
caso, como desgraciadamente no disponemos del tiempo para poder usar todos los métodos, hemos realizado las siguientes.

\begin{itemize}
    \item \textbf{Entrevistas:} Es una de las partes más importantes de la investigación. \textbf{CONTINUAR}
    \item \textbf{ESTO CREO QUE MEJOR HACERLO AL FINAL PARA VER LO QUE TENEMOS HECHO}
\end{itemize}



Pero antes de realizar esta labor, debemos saber reconocer cuál es el público objetivo de nuestra aplicación y
de qué manera podemos clasificar a los distintos perfiles dentro de los clientes potenciales. Para eso hemos realizado la
\textbf{Hipótesis de personas}.

\section{Hipótesis de personas}

Para poder hacer correctamente las labores de investigación, es muy importante saber discernir entre potenciales clientes y usuarios
a los que no va dirigida (en principio) la aplicación. Hemos analizado nuestros objetivos con \textit{<NombreAplicación>}, y hemos hecho las siguientes
diferenciaciones:

\begin{itemize}
    \item \textbf{Personas con \gls{di}:} Obviamente, este es nuestro público principal. Son las personas en las que pensamos cuando
            elegimos diseñar esta aplicación. Pero dentro de esta categoría podemos subcategorizar a los individuos:
            \begin{itemize}
                \item \textbf{\gls{di} leve:} La mayoría de personas con \gls{di} pertenecen a este grupo, y son en las que más nos vamos a centrar.
                        Esto es debido a que en su mayoría tienen un mayor nivel de independencia, por lo que será más fácil que quieran hacer algún viaje, ya
                        sea en solitario o en compañía.
                \item \textbf{\gls{di} moderado o grave:} En estos casos, deberíamos considerar la dependencia de la persona. Es menos probable que estos individuos
                        se metan en la página, ya que por lo general disponen de un tutor o alguien a cargo que será el que lo organice en caso de viajer. En
                        este caso, el tutor sería el cliente potencial, ya que es el que usaría la aplicación. 
            \end{itemize}
    \item \textbf{Acompañantes de personas con \gls{di}:} También es un público importante de nuestra aplicación. Ésto es debido a que en muchas casos las personas con
            D.I. será tan dependiente que necesitará de otra persona para poder realizar la búsqueda. Éstas personas pueden enfrentarse a distintos problemas
            a la hora de reser var que habrá que tener en cuenta.
\end{itemize}
