\chapterA{Introducción}

\textit{<NombreAplición>} ha surgido como proyecto debido a la necesidad de viajar que tienen las distintas personas con discapacidad intelectual. Al
descubrir que muchas de éstas tienen problemas a la hora de reservar sus viajes con las páginas actualmente disponibles, decidimos crear un 
comparador de viajes que sean capaces de usar las personas con una limitación intelectual leve, o sus acompañantes en el caso de personas que
tengan una discapacidad mayor. 

Para lograr esto, vamos a empezar realizando una investigación sobre los usuarios potenciales y sus necesidades. Para esto realizaremos entrevistas
a distintos perfiles dentro de nuestras hipótesis de personas, para así poder diseñar la aplicación en referencia a sus experiencias y problemas a la hora
de viajar y/o buscar viajes. Junto a las entrevistas, también se realizará la observación de los usuarios en ámbitos de búsqueda de viajes, para poder hacer
una observación de éste. Además, para alcanzar una mayor demografía, haremos uso de cuestionarios, los cuáles nos permitirán también recibir información
en menor medida que en una entrevista, pero de más gente.

Después de la investigación, realizaremos un modelado de la aplicación, utilizando los resultados obtenidos para crear arquetipos de usuarios que
contengan información sobre los distintos objetivos que tendrían los perfiles potenciales. Gracias a esto, podríamos obtener usuarios que nos den
\textit{feedback} a la hora de avanzar con el diseño del sistema.

