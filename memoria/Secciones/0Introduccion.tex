\chapterA{Introducción}

\textit{\app} surge en un primer momento como idea de aplicación para ayudar a la gente con Discapacidad Intelectual a tener menos problemas a la hora de comparar viajes.
Tras entrevistar a unas cuantas personas con discapacidad, vemos que realmente no hay tanto mercado para este tipo de aplicación ya que muchos de ellos o no organizaban los viajes o no encontraban grandes problemas con los comparadores actuales.
Esta conclusión nos hizo tomar la decisión de cambiar el proyecto ligeramente para hacer un comparador de viajes para todo el mundo, mejorando los que hay actualmente en el mercado, pero manteniendo las mejoras para gente con discapacidad.

Algunos de los problemas de los comparadores modernos son sus limitaciones a la hora de comparar viajes completos (medio de transporte y alojamiento), interfaces complicadas y/o sobrecargadas y falta de características para gente con discapacidad.

Para resolver este problema haremos uso de un diseño de interfaces moderno y simplista, quitando elementos que puedan dar lugar a confusiones y dejando solo las partes esenciales. Al eliminar elementos innecesarios podremos añadir nuevas funcionalidades y adaptaciones para que nuestra aplicación pueda ser usada por todo el mundo, desde jóvenes adultos a personas mayores y gente con y sin discapacidades.

Después de la investigación, realizaremos un modelado de la aplicación, utilizando los resultados obtenidos para crear arquetipos de usuarios que
contengan información sobre los distintos objetivos que tendrían los perfiles potenciales. Gracias a esto, podríamos obtener usuarios que nos den
\textit{feedback} a la hora de avanzar con el diseño del sistema.
