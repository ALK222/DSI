\section{Búsqueda y Selección de Participantes}
Tras haber finalizado el plan de evaluación, en el que se definieron las tareas iniciales que se iban a realizar (entre las que se
encuentra los requisitos que van a tener los usuarios participantes), vamos a realizar un proceso de búsqueda de usuarios finales 
para poder realizar esta evaluación con usuarios. \\

Como ya vimos, uno de los requisitos que necesitamos (y que es inherente a nuestras personas) es que que les guste viajar, que comprendan
y tengan un buen manejo de la tecnología y usen aplicaciones de este tipo (o similares) para poder reservar sus viajes 
o poder encontrar el mejor precio posible. \\

Una vez definidos los requisitos básicos que tienen que tener los usuarios, vamos a proceder a buscar usuarios potenciales que puedan servirnos
para realizar estas evaluaciones con usuarios finales. Estos usuarios que hemos seleccionado son amigos, conocidos y familiares nuestros pero que no hayan
participado en la fase de investigación y son:
\begin{itemize}
    \item Alejandro Rivera
    \item Ángel Toriello
    \item Jorge Marinas
    \item Marta Valle Inclán
    \item Pablo Sánchez
    \item Silvia Gómez
\end{itemize}

Tras tener este primer listado de usuarios, les hemos facilitado un cuestionario inicial\footnote{\url{https://docs.google.com/forms/d/19Q6juq5hT84P_bZL0rkkHTSv5pXL463pXtBPqb7H3uE/edit}} para conocerlos un poco más en profundidad y ver si sus características
coinciden con las requeridas para poder realizar estas evaluaciones. Este cuestionario de screening o previo nos ha ayudado a asegurarnos de que los usuarios seleccionados representaban a un usuario potencial.
En este cuestionario inicial, la sección que determina si el usuario es potencial o no es la general, en la cual se pregunta acerca de la edad, género y las 
preferencias de viaje. Las preferencias de viaje son cruciales para concluir si el usuario será entrevistado o no y además de entender mejor las motivaciones 
de los usuarios a la hora de viajar, si disfrutan o no o la frecuencia con la que viajan, entre otras. \\

Además, también se ha hecho una clasificación de usuario con discapacidad, acompañante o sin discapacidad para conocer mejor el nivel del usuario, ya que cabe recordar 
que la aplicación ofrece una opciones de accesibilidad. Cabe destacar que no hemos creado un cuestionario previo, ya que hemos decidido que sea el mismo tanto el de 
screening como el previo. \\ 

Analizando los resultados, todos los usuarios eran compatibles con los perfiles buscados (aunque finalmente el usuario Jorge tuvimos que descartarlo por incompatibilidad
de horarios). Por tanto, el listado definitivo de usuarios a entrevistar quedaría así:
\begin{itemize}
    \item Alejandro Rivera
    \item Ángel Toriello
    \item Jorge Marinas
    \item Marta Valle Inclán
    \item Pablo Sánchez
    \item Silvia Gómez
\end{itemize}

Todos estos usuarios se corresponden con la persona de Marta, ya que son personas jóvenes (estudiantes) a las que les gusta viajar por ocio y no presentan ninguna discapacidad
(como era el caso de la persona de Isabel). \\

Finalmente, estos usuarios serán los que participarán en la fase de evaluación (grabada). Para solicitar este consentimiento de grabación, antes de confirmarles que 
habían sido seleccionados para la entrevista se les informó que iba a ser grabada para poder ser estudiada y analizada con la finalidad de obtener los resultados
pertinentes y que estamos intentando obtener (como vimos en el plan de evaluación).