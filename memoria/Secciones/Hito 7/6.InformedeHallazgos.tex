\section{Informe de hallazgos y recomendaciones}
Una vez analizados los datos y obtenidos los principales problemas, vamos a proceder a realizar el informe de hallazgos y recomendaciones \footnote{\url{https://docs.google.com/spreadsheets/d/1TxDvMkI0_I-ulgKDzlJs5LAwtKchI6tg8Bn9Vx-8N3w/edit?usp=sharing}}. Este informe consistirá
en un listado de todos los problemas principales que se han podido identificar, así como una descripción del problema, la pantalla en la que se puede reproducir,
una posible solución que pueda tener y finalmente un cálculo de la prioridad que tiene su resolución. \\

Para poder calcular la prioridad, hemos necesitado los valores de severidad y coste, asignados por nosotros en función de cómo consideramos sus características. Vamos
a emplear la severidad siendo 4 una catástrofe y 1 una severidad mínima, así como para el coste un 1 nos supondrá un coste excesivo y un 4 un coste ínfimo. Para calcular
la prioridad empleamos la fórmula \textit{prioridad = severidad * coste}, obteniendo así el listado completo de todos los problemas identificados. \\

A modo de pensamiento futuro, nuestro prototipo podría ser mejorado mediante la solución de estos problemas (recogidos en el informe de hallazgos y recomendaciones) así
como los propuestos en la lista final de problemas y soluciones elaborada en el hito anterior fruto de la evaluación con los expertos. Podemos observar que la cantidad de
problemas detectados por los usuarios expertos es mucho mayor (debido al uso de las heurísticas) que la realizada por los usuarios finales (únicamente poseen su criterio propio).

