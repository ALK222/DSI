\section{Introducción}
Tras haber finalizado la fase de evaluación heurística, el siguiente paso que tenemos que realizar es una evaluación con usuarios finales. Esta nueva evaluación nos dará un
tipo de \textit{feedback} más completo y sin necesidad de estar atado a ningún principio. Los pasos que han de seguirse para poder realizar esta etapa son los siguientes:
\begin{itemize}
    \item \textbf{Plan de evaluación} $\rightarrow$ el plan de evaluación consiste en una preparación inicial de las actividades que van a realizarse durante todo el proceso
    de evaluación, estas tareas son las siguientes:
    \begin{itemize}
        \item \textit{Propósito y objetivos de la evaluación} $\rightarrow$ antes de comenzar la evaluación, es necesario clarificar el propósito por el que va a realizarse,
        así como los objetivos que han de cumplirse al final del transcurso de la misma.
        \item \textit{Formular las preguntas de investigación} $\rightarrow$ el siguiente de los pasos que han de seguirse es plantear una serie de preguntas de investigación,
        siendo nuestro objetivo poder darle a cada una de ellas una respuesta al final de esta evaluación.
        \item \textit{Identificar los requisitos para los participantes} $\rightarrow$ se han identificado una serie de características que han de tener los usuarios finales 
        que van a realizar la prueba de nuestra aplicación para poder obtener el \textit{feedback} necesario.
        \item \textit{Describir el diseño experimental} $\rightarrow$ describe el diseño experimental de la aplicación que contiene el contexto de la evaluación, el rol
        que va a tener el moderador y el análisis que se va a llevar a cabo de los datos. Esta descripción englobará los siguientes puntos:
        \begin{itemize}
            \item Listado de tareas a realizar. Lista de las tareas que van a realizarse en las pruebas guiadas, que cubren los casos planteados en los escenarios keypath
            y en los escenarios de validación.
            \item Descripción del entorno y las herramientas que se van a utilizar. Breve descripción informativa del material que va a emplearse para cada una de las sesiones
            de evaluación.
            \item Tareas del moderador. Definición de las funciones que realizará el moderador durante cada una de las sesiones de evaluación que tendrán lugar.
            \item Identificación de los datos que se van a recolectar. Se discutirán aquellos datos que se considere necesario que se vayan a utilizar y se elaborará una
            plantilla sencilla para poder recogerlos. Asimismo, también se realizarán los cuestionarios necesarios para las distintas sesiones de evaluación (tanto el de inicio como
            el de fin).
            \item Descripción de la metodología de análisis de datos. Se describirá de forma aproximada (ya que nos encontramos en una fase de trabajo muy temprana) el plan que
            va a seguirse para obtener un análisis de los datos.
        \end{itemize}
    \end{itemize}
    \item \textbf{Búsqueda y selección de participantes} $\rightarrow$ se realizará una búsqueda de distintos usuarios finales de nuestra aplicación que cumplan los requisitos que hemos
    mencionado anteriormente. Asimismo, deberán cumplimentar un cuestionario inicial en el que se comprobará si son usuarios aptos para nuestra evaluación.
    \item \textbf{Sesiones de evaluación} $\rightarrow$ sesiones de evaluación con los distintos usuarios finales que previamente hemos seleccionado. En estas sesiones extraeremos
    los datos que habíamos planteado.
    \item \textbf{Análisis de los datos obtenidos} $\rightarrow$ tras haber finalizado las sesiones de evaluación que habían sido planteadas y extraídos los datos necesarios, se procede a
    realizar un análisis (siguiendo la metodología descrita) de los mismos.
    \item \textbf{Informe de hallazgos y recomendaciones} $\rightarrow$ se realizará un informe de los hallazgos y recomendaciones, recogiendo todos los problemas que se han identificado, así
    como la severidad y el coste.
    \item \textbf{Respuestas a las preguntas de investigación} $\rightarrow$ a modo de conclusión de este hito, con los resultados obtenidos de las distintas evaluaciones, se dará una respuesta
    basada en evidencias a estas preguntas que han sido planteadas al principio.
\end{itemize}