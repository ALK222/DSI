\section{Análisis de los datos obtenidos}
Una vez transcurridas las sesiones de evaluación, hemos podido recopilar todo el material necesario para poder comenzar
con la fase de análisis. En esta fase vamos a tomar los datos obtenidos: los cuestionarios finales (realizados posterior a
la sesión de evaluación) y las anotaciones realizadas por los observadores de las sesiones en las plantillas para poder obtener
una idea mucho más clara de los principales problemas que presenta nuestra aplicación. El objetivo es obtener un listado de los
principales problemas que se han encontrado en la aplicación para desarrollarlos en la siguiente fase \\

En primer lugar, vamos a proceder a realizar un análisis de las tablas rellenas por los distintos observadores de las sesiones
y en las que se recoge el transcurso de las distintas entrevistas, observando y anotando las sensaciones del usuario y los comentarios
para cada una de las actividades que le han sido planteadas. Vamos a proceder a comentar cada una de las distintas anotaciones de las sesiones,
obteniendo los aspectos más destacados y posteriormente realizar una conclusión.
\begin{itemize}
    \item \textbf{Sesión 1 - Marta} $\rightarrow$ Marta ha realizado sin ningún problema las tareas 1 - 6, 10 y 14 - 18. Por otro lado, ha presentado
          pequeñas confusiones en las tareas 7, 9 y 11, debido a confusión de botones (filtrar con vuelos), deshacer el filtro (le ha supuesto casi un problema)
          y el botón de viaje accesible(le ha parecido poco intuitivo). Finalmente, la tarea que más frustración le ha causado ha sido la número 13, ya que no
          ha podido localizar el botón de preguntas frecuentes de forma intuitiva.
    \item \textbf{Sesión 2 - Ángel} $\rightarrow$ Ángel se ha mostrado tranquilo durante todo el transcurso de la entrevista y ha podido realizar todas las
          tareas planteadas sin encontrar problema alguno. El único obstáculo que ha tenido a sido una pequeña confusión para obtener los detalles del viaje, ya que
          pensaba que tenía que pulsar sobre la tarjeta completa (y no sobre el botón de información).
    \item \textbf{Sesión 3 - Pablo} $\rightarrow$ Pablo, al igual que Ángel en la sesión anterior, se ha mostrado tranquilo durante toda la entrevista, pero ha tenido
          más momentos de confusión (sobre todo en las actividades 8, 9, 12 y 17). En ellas, los problemas que ha identificado, respectivamente son: problemas al buscar los
          detalles de un viaje, seleccionar asientos antes de rellenar los pasajeros (le ha confundido ese orden), volver a la página de inicio desde donde se encontraba y
          al cancelar ha buscado en lugar de en la reserva en la opción de modificar reserva.
    \item \textbf{Sesión 4 - Alejandro} $\rightarrow$ Alejandro no ha tenido, al igual que en los casos anteriores, ningún problema en la mayoría de las actividades, pero
          sí que ha experimentado momentos de confusión en las tareas 7, 8, 12, 17 y 18, debido a que ha pensado que los filtros de transporte se encontraban en la sección de filtros,
          no ha podido identificar el botón de detalles de las tarjetas de viajes al realizar una búsqueda, ha tenido que seleccionar los asientos antes de dar los datos de los
          pasajeros y para cancelar una reserva se ha metido directamente en consultar reserva.
    \item \textbf{Sesión 5 - Silvia} $\rightarrow$ Silvia se ha mostrado tranquila en todo momento y ha podido resolver la gran mayoría de las tareas sin ningún problema,
          aunque ha presentado momentos de confusión en las tareas 8, 11 y 12 al inicio de las mismas, debido a problemas con la identificación del botón de detalles, desconocimiento del
          concepto de accesibilidad (no entiende a lo que se refiere la tarea) y algunas complicaciones experimentadas al realizar el pago de la reserva efectuada. Por último, y a
          diferencia del resto de sesiones, ha tenido que solicitar ayuda para poder completar una de las tareas (la número 15), ya que no sabía dónde podía encontrarla.
\end{itemize}

Tras haber finalizado este análisis de las distintas evaluaciones realizadas por los usuarios, hemos denotado que la mayoría de ellos han presentado problemas en una serie
de tareas (la 8, la 11, la 12 y la 17), debido a factores que provienen de la realización de esta actividad y que han supuesto momentos de confusión para los usuarios. Estos problemas,
de forma breve son los siguientes:
\begin{itemize}
    \item \textit{Actividad 8} $\rightarrow$ el icono para poder encontrar y consultar los detalles del viaje no se muestra de forma clara para los usuarios, que ven de una forma mucho
          más intuitiva y natural pulsar sobre la propia tarjeta (que sirve para seleccionar el viaje) que acceder a través de este icono.
    \item \textit{Actividad 11} $\rightarrow$ la aplicación te pide que selecciones (dentro de la misma pantalla) los asientos antes de poder rellenar los datos del formulario para cada
          uno de los pasajeros (algunos usuarios han tenido problemas ya que lo primero que iban a hacer era rellenar sus datos).
    \item \textit{Actividad 12} $\rightarrow$ problemas para volver a la página de inicio en algunos casos y también complicaciones para poder elegir el método de pago.
    \item \textit{Actividad 17} $\rightarrow$ la mayoría de los usuarios ha tenido problemas en esta actividad, ya que han supuesto que la ubicación del botón de cancelación
          se encontraba o bien en el interior de los datos de la reserva o bien el la opción de modificar los datos.
    \item Aunque no lo se ha ligado específicamente a la actividad, algunos de los usuarios han tenido problemas con la ubicación de las preguntas frecuentes, ya que consideraban que ese no era
          el lugar correcto (no es intuitivo), así como que su tamaño es demasiado pequeño y en algunos casos puede no ser apreciado correctamente.
\end{itemize}

El cuestionario de final de entrevista se mandó a las 5 personas entrevistadas, pero solo una de ellas (Pablo) contestó, por lo que el análisis de este cuestionario no va a ser muy relevante ya que solo aporta el punto de vista de una persona.

En cuanto a satisfacción general, vemos que el usuario se ha sentido bastante cómodo con la herramienta, destacando sencillez y comodidad en su uso y destacando que no hace falta mucha experiencia con la misma para utilizarla.
Entrando en la parte de SUS, la primera sección sería ``Evaluación de la utilidad del sistema'', en la cual conseguimos 16, siendo mejorable el apartado de mostrar mensajes más claros para los problemas.
En la sección ``Evaluación de la calidad del sistema'' sacamos una puntuación de 17, solo consiguiendo una nota perfecta en la claridad de la información aportada por el sistema.
La última sección ``Evaluación de la calidad de la interfaz'', nos da una puntuación perfecta de 20. Con estos datos, al multiplicar cada puntuación por 2.5 y sumarlas todas, conseguimos una puntuación final de 132,5. Con esta puntuación, basándonos en los cálculos hechos en el apartado 2, obtenemos una puntuación bastante por encima de la media (aunque esta solo sea la opinión de un usuario).

De este cuestionario salimos con un problema claro: la información podría ser ligeramente más clara, pero bastante más precisa en el apartado de ayudas para solucionar errores.