\section{Sesiones de evaluación}

\subsection{Marta}

La primera entrevista fue realizado por Pablo y Carlos a Marta, amiga de Pablo, el día 23 de diciembre de 2023. La entrevista
comienza con Pablo agradeciendo a Marta su colaboración en la evaluación y pidiéndole permiso para realizar la grabación. Marta
contesta que sí, se le explica el objetivo de nuestra aplicación y comienza el estudio.

\begin{enumerate}
    \item En la primera tarea, la de registrarse, comenta que es "súper intuitivo" y la puede hacer sin problemas.
    \item Al cerrar sesión tampoco tiene ningún problema realiza la tarea fácilmente.
    \item El inicio de sesión lo realiza de una manera rápida y sin complicaciones.
    \item Cuando se le pide que cambie el DNI del usuario vemos que accede correctamente a la pantalla de modificar perfil
        pero una vez dentro se queda buscando donde está la opción del DNI. Esto sucede muy poco, apenas unos segundos, pero
        por la situación del cursor podemos pensar que incialmente lo buscaba en la zona de los datos personales como el nombre, en la
        parte superior, y no en la parte inferior derecha de la ventana. Una vez encontrado no tiene problemas para realizar el cambio
        del DNI (no se muestra ya que no está implementado en \textit{Figma}).
    \item No tiene ningún problema para volver al la página principal desde el perfil. Cabe destacar que usa el botón de marcado
        como "atrás", y no pulsa el logotipo para volver a la pantalla principal, la otra opción para realizar esta tarea.
    \item No tiene ningún problema en realizar la búsqueda de los viajes.
    \item Esta tarea es la que más le cuesta hasta el momento. Tarda bastante tiempo en encontrar los filtros, y comienza buscando
        en el apartado de "Filtros", pero el filtro por transporte está separado del resto. Al final lo encuentra y puede finalizar la
        tarea correctamente.
    \item Ésta es la primera tarea que no consigue completar, ya que Pablo tiene que decirle que es lo que tiene que pulsar, tras un tiempo
        de búsqueda. Su primera intención era pasar a la siguiente pantalla, con la idea de poder observar la información ahí, pero no se le
        ocurrió pulsar el botón designado para la información.
    \item Este filtro en cambio sí que lo encuentra fácilmente y lo activa, el problema es al desactivarlo, que se equivoca y le da al botón de
        volver atrás. Luego lo reintenta y lo hace bien, pero comenta que es algo confuso lo de los botones de "Atrás" y "Siguiente".
    \item Ningún problema en esta tarea.
    \item En esta tarea encuentra fácilmente como hacerlo en los filtros, pero desde la página principal no ve el \textit{checkbox} de viaje
        accesible, por lo que no lo pulsa.
    \item Al realizar la selección de los viajes de ida y vuelta, simplemente selecciona el de ida y va a pasar a la siguiente página sin seleccionar
        el de vuelta, pensando que con seleccionar uno había seleccionado los dos trayectos. En la siguiente página, se le olvida que debe seleccionar
        los asientos, pero cuando Pablo se lo recuerda puede hacerlo sin problemas. Al realizar el pago no tiene ningún problema y puede terminar
        la reserva correctamente.
    \item En esta tarea no tiene complicaciones para llegar a la sección de mis reservas, pero luego dentro le cuesta un poco encontrar el botón en el que
        se puede acceder a descargar los billetes. De hecho tiene que pulsar en la pantalla para que \textit{Figma} le marque los botones que hay disponibles
        en la pantalla. A pesar de esto, tampoco tarda mucho en localizarlos y consigue completar la tarea. Al final de la tarea comenta el problema que ha tenido
        con los iconos de la página de mis reservas, diciendo que son "poco claros".
    \item Esta tarea le cuesta bastante, ya que tiene que buscar el botón de \textit{F.A.Q.} por varias pantallas, a pesar de que se
        encuentra en todas. Hay un momento que hasta pregunta dónde se encuentra, pero consigue encontrarlo antes de que los evaluadores
        respondan. Comenta que a lo mejor pondría los iconos de ayuda al lado del perfil.
    \item Gracias a la tarea anterior identificó que el icono de servicio al cliente era el que estaba al lado de preguntas frecuentes, por lo
        supo completar esta tarea sin problemas.
    \item No tiene problemas realizando esta tarea.
    \item No ha habido problemas al cancelar el viaje. Ha habido un momento de pausa, quizá porque estaba pensando en si pulsando la cruz roja eliminaría
        a ambos pasajeros, pero finalmente pulsa y ve que se puede seleccionar el pasajero a borrar.
    \item Para esta tarea no tiene problema, ya que es muy parecida a la anterior.
\end{enumerate}

A continuación se hacen a Marta las preguntas que habíamos definido anteriormente para comprobar la opinión de Marta:

\begin{itemize}
    \item \textbf{?`Sabes qué elementos son accionables?} Sí. 
    \item \textbf{?`Es fácil completar el proceso de registro?} Sí. 
    \item \textbf{?`Has conseguido encontrar el contenido que buscabas?} Sí.
    \item \textbf{?`Cambias fácilmente de una pantalla a otra?} Sí.
    \item \textbf{?`Se entienden los iconos y los símbolos de la aplicación?} Dice que modificaría los de descarga de billetes y propone
        también añadir texto explicativo al pasar el ratón por encima. Pregunta Carlos que si cambiaría los de soporte y preguntas frecuentes
        y dice que se ven bien pero que a lo mejor los pondría junto al perfil.
    \item \textbf{?`Se pueden hacer las acciones importantes con pocas pulsaciones?} Sí.
    \item \textbf{?`Es cómodo accionar los controles?} Sí.
    \item \textbf{?`Los flujos de aplicación son fáciles de entender?} Sí.
    \item \textbf{?`Los usuarios siempre encuentran los controles que necesitan?} Sí.
\end{itemize}