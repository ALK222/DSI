\section{Sesiones de evaluación}

Tras realizar la selección de individuos en el apartado anterior, se ha procedido a realizar la evaluación de los individuos. Explicamos
su transcurso y las conclusiones a continuación

\subsection[Marta]{Marta\footnote{Vídeo con la entrevista a Marta: \url{https://drive.google.com/file/d/1n-NszrENiD-pDnbTnnsufwc15JMou5r-/view?usp=drive_link}}}

La primera entrevista fue realizado por Pablo y Carlos a Marta, amiga de Pablo, el día 23 de diciembre de 2023. La entrevista
comienza con Pablo agradeciendo a Marta su colaboración en la evaluación y pidiéndole permiso para realizar la grabación. Marta
contesta que sí, se le explica el objetivo de nuestra aplicación y comienza el estudio.

\begin{enumerate}
    \item En la primera tarea, la de registrarse, comenta que es ``súper intuitivo'' y la puede hacer sin problemas.
    \item Al cerrar sesión tampoco tiene ningún problema realiza la tarea fácilmente.
    \item El inicio de sesión lo realiza de una manera rápida y sin complicaciones.
    \item Cuando se le pide que cambie el DNI del usuario vemos que accede correctamente a la pantalla de modificar perfil
        pero una vez dentro se queda buscando donde está la opción del DNI. Esto sucede muy poco, apenas unos segundos, pero
        por la situación del cursor podemos pensar que incialmente lo buscaba en la zona de los datos personales como el nombre, en la
        parte superior, y no en la parte inferior derecha de la ventana. Una vez encontrado no tiene problemas para realizar el cambio
        del DNI (no se muestra ya que no está implementado en \textit{Figma}).
    \item No tiene ningún problema para volver al la página principal desde el perfil. Cabe destacar que usa el botón de marcado
        como ``atrás'', y no pulsa el logotipo para volver a la pantalla principal, la otra opción para realizar esta tarea.
    \item No tiene ningún problema en realizar la búsqueda de los viajes.
    \item Esta tarea es la que más le cuesta hasta el momento. Tarda bastante tiempo en encontrar los filtros, y comienza buscando
        en el apartado de ``Filtros'', pero el filtro por transporte está separado del resto. Al final lo encuentra y puede finalizar la
        tarea correctamente.
    \item Ésta es la primera tarea que no consigue completar, ya que Pablo tiene que decirle que es lo que tiene que pulsar, tras un tiempo
        de búsqueda. Su primera intención era pasar a la siguiente pantalla, con la idea de poder observar la información ahí, pero no se le
        ocurrió pulsar el botón designado para la información.
    \item Este filtro en cambio sí que lo encuentra fácilmente y lo activa, el problema es al desactivarlo, que se equivoca y le da al botón de
        volver atrás. Luego lo reintenta y lo hace bien, pero comenta que es algo confuso lo de los botones de ``Atrás'' y ``Siguiente''.
    \item Ningún problema en esta tarea.
    \item En esta tarea encuentra fácilmente como hacerlo en los filtros, pero desde la página principal no ve el \textit{checkbox} de viaje
        accesible, por lo que no lo pulsa.
    \item Al realizar la selección de los viajes de ida y vuelta, simplemente selecciona el de ida y va a pasar a la siguiente página sin seleccionar
        el de vuelta, pensando que con seleccionar uno había seleccionado los dos trayectos. En la siguiente página, se le olvida que debe seleccionar
        los asientos, pero cuando Pablo se lo recuerda puede hacerlo sin problemas. Al realizar el pago no tiene ningún problema y puede terminar
        la reserva correctamente.
    \item En esta tarea no tiene complicaciones para llegar a la sección de mis reservas, pero luego dentro le cuesta un poco encontrar el botón en el que
        se puede acceder a descargar los billetes. De hecho tiene que pulsar en la pantalla para que \textit{Figma} le marque los botones que hay disponibles
        en la pantalla. A pesar de esto, tampoco tarda mucho en localizarlos y consigue completar la tarea. Al final de la tarea comenta el problema que ha tenido
        con los iconos de la página de mis reservas, diciendo que son ``poco claros''.
    \item Esta tarea le cuesta bastante, ya que tiene que buscar el botón de \textit{F.A.Q.} por varias pantallas, a pesar de que se
        encuentra en todas. Hay un momento que hasta pregunta dónde se encuentra, pero consigue encontrarlo antes de que los evaluadores
        respondan. Comenta que a lo mejor pondría los iconos de ayuda al lado del perfil.
    \item Gracias a la tarea anterior identificó que el icono de servicio al cliente era el que estaba al lado de preguntas frecuentes, por lo
        supo completar esta tarea sin problemas.
    \item No tiene problemas realizando esta tarea.
    \item No ha habido problemas al cancelar el viaje. Ha habido un momento de pausa, quizá porque estaba pensando en si pulsando la cruz roja eliminaría
        a ambos pasajeros, pero finalmente pulsa y ve que se puede seleccionar el pasajero a borrar.
    \item Para esta tarea no tiene problema, ya que es muy parecida a la anterior.
\end{enumerate}

A continuación se hacen a Marta las preguntas que habíamos definido anteriormente para comprobar su opinión sobre nuestra aplicación:

\begin{itemize}
    \item \textbf{?`Sabes qué elementos son accionables?} Sí. 
    \item \textbf{?`Es fácil completar el proceso de registro?} Sí. 
    \item \textbf{?`Has conseguido encontrar el contenido que buscabas?} Sí.
    \item \textbf{?`Cambias fácilmente de una pantalla a otra?} Sí.
    \item \textbf{?`Se entienden los iconos y los símbolos de la aplicación?} Dice que modificaría los de descarga de billetes y propone
        también añadir texto explicativo al pasar el ratón por encima. Pregunta Carlos que si cambiaría los de soporte y preguntas frecuentes
        y dice que se ven bien pero que a lo mejor los pondría junto al perfil.
    \item \textbf{?`Se pueden hacer las acciones importantes con pocas pulsaciones?} Sí.
    \item \textbf{?`Es cómodo accionar los controles?} Sí.
    \item \textbf{?`Los flujos de aplicación son fáciles de entender?} Sí.
    \item \textbf{?`Los usuarios siempre encuentran los controles que necesitan?} Sí.
\end{itemize}

Los datos recogidos se han plasmado en una tabla\footnote{Accesible a través de:
 \url{https://docs.google.com/spreadsheets/d/1jzu5b4KD2aB9WzM2ZhH__ciSOqkHw-PuBl2CbRHKKAI/edit?usp=drive_link}}.

\subsection[Ángel]{Ángel\footnote{Vídeo con la entrevista a Ángel: \url{https://drive.google.com/file/d/18542RuuM21HuHkiFVz0Z-WGK2TxB2amf/view?usp=drive_link}}}

La segunda entrevista fue realizado por Pablo y Carlos a Ángel, amigo de Pablo, el día 25 de diciembre de 2023. La entrevista
comienza con Pablo agradeciendo a Ángel su colaboración en la evaluación y pidiéndole permiso para realizar la grabación. Ángel
acepta, se le explica el objetivo de nuestra aplicación y comienza el estudio.

\begin{enumerate}
    \item No tiene problemas con esta tarea.
    \item Tampoco tiene problemas realizando esta tarea.
    \item Ningún problema iniciando sesión.
    \item Ningún problema.
    \item Vuelve incluso antes de que Pablo le diga la tarea, por lo que no tiene problemas.
    \item Ningún problema buscando el viaje. Le pregunta Pablo que qué le parece la ventan y Ángel comenta que le sobra la columna
        que hay a la izquierda (la usada para el filtrado).
    \item Tarda un poco más que en el resto de tareas en encontrar el botón, probablemente por buscar en la zona de filtros primero.
    \item Pulsa inicialmente en la selección del viaje, pero rápidamente se da cuenta del botón de información y lo pulsa.
    \item Ningún problema en esta tarea.
    \item Tampoco tiene problemas realizando la ordenación.
    \item No tiene problemas, ocurre un error del prototipo pero Pablo interviene rápidamente y finaliza la tarea correctamente.
    \item Igual que antes ocurre algún fallo con el prototipo al elegir los asientos, pero intervienen los evaluadores y consigue terminar
        la tarea fácilmente.
    \item Consigue hacerlo sin problemas.
    \item Tampoco tiene problemas usando el chat de la atención al cliente.
    \item No tiene problemas para acceder a la sección de mis reservas, pero luego le cuesta un poco encontrar el botón en el que se puede acceder
        a la descarga de los billetes.
    \item Igual que en el anterior, una vez en mis reservas le cuesta encotrar el icono, pero tampoco tarda mucho, apenas unos segundos.
    \item No tiene problemas cancelando el viaje para un pasajero.
    \item Tampoco los tiene cancelando todo el viaje
\end{enumerate}

A continuación se hacen a Ángel las preguntas que habíamos definido anteriormente para comprobar su opinión sobre nuestra aplicación:

\begin{itemize}
    \item \textbf{?`Sabes qué elementos son accionables?} Sí. 
    \item \textbf{?`Es fácil completar el proceso de registro?} Sí. 
    \item \textbf{?`Has conseguido encontrar el contenido que buscabas?} Sí.
    \item \textbf{?`Cambias fácilmente de una pantalla a otra?} Sí.
    \item \textbf{?`Se entienden los iconos y los símbolos de la aplicación?} Sí
    \item \textbf{?`Se pueden hacer las acciones importantes con pocas pulsaciones?} Sí.
    \item \textbf{?`Es cómodo accionar los controles?} Sí.
    \item \textbf{?`Los flujos de aplicación son fáciles de entender?} Sí.
    \item \textbf{?`Los usuarios siempre encuentran los controles que necesitan?} Sí.
\end{itemize}

Los datos recogidos se han plasmado en una tabla\footnote{Accesible a través de:
 \url{https://docs.google.com/spreadsheets/d/1u1Hz8r84sr9u3jgAo186t4P0KIgxSTTdyt3XcVZTENc/edit?usp=drive_link}}.

\subsection[Pablo]{Pablo\footnote{Vídeo con la entrevista a Pablo: \url{https://drive.google.com/file/d/1J2M8x856G_Qt9Thh-2af4mrGmhjQvJA8/view?usp=drive_link}}}

La tercera entrevista fue realizada por Alejandro y María a Pablo, amigo de Alejandro, el día 27 de diciembre de 2023. La entrevista comienza con Alejandro agradeciendo a Pablo su colaboración en la evaluación y pidiéndole permiso para realizar la grabación. Pablo contesta que sí, se le explica el objetivo de nuestra aplicación y comienza el estudio.

\begin{enumerate}
    \item Hay bastante información en la pantalla de registro y son demasiados campos y muy juntos. Pero le resulta más o menos intuitivo.
    \item Le ha resultado bastante intuitivo.
    \item Ningún problema.
    \item Le ha resultado bastante intuitivo.
    \item Lo ha hecho sin que se lo dijeran.
    \item Ningún problema salvo que Alejandro le ha tenido que comentar que solo ha rellenado los campos para realizar la búsqueda y que tiene que darle al botón de buscar.
    \item Lo ha realizado sin complicaciones.
    \item Le ha costado mucho descubrir cómo ver los detalles de un viaje de ida. Comenta que el botón de información es muy pequeño y poco vistoso. Y supone mayor dificultad visualizarlo cuando el viaje ya ha sido seleccionado.
    \item Se ha confundido y ha ido para filtrar por precio al botón de ordenar pero comenta que ha sido una confusión suya, no de la aplicación. Alejandro se lo comenta debido a que usuarios anteriores se han quejado de ello, pero Pablo dice que tiene sentido separar filtrar y ordenar.
    \item Le parece bien a excepción de deshacer la ordenación de resultados puesto que tener que darle al botón de atrás Pablo piensa que es para volver a la pantalla anterior.
    \item Ningún problema.
    \item Se ha sentido confuso debido a las limitaciones de Figma, pero lo ha podido hacer sin problema.
    \item Ningún problema con consultar preguntas frecuentes.
    \item Ningún problema con el chat.
    \item No ha tenido problemas para consultar los datos de la reserva y descargarlos.
    \item Ningún problema para modificar los servicios adicionales.
    \item Le ha costado encontrar cómo cancelar la reserva de solo el segundo viajero porque pensaba que era para cancelar el viaje completo.
    \item No ha tenido problema y ha intuido que se hacía cancelando el viaje de todos los viajeros.
\end{enumerate}

Se finaliza la entrevista dando algún feedback más pero se les olvidó comentar las preguntas que finalizan la entrevista. Pero acorde con la observación de Pablo se pueden responder y son las siguientes respuestas:

\begin{itemize}
    \item \textbf{?`Sabes qué elementos son accionables?} Sí. 
    \item \textbf{?`Es fácil completar el proceso de registro?} Sí. 
    \item \textbf{?`Has conseguido encontrar el contenido que buscabas?} Sí.
    \item \textbf{?`Cambias fácilmente de una pantalla a otra?} Sí.
    \item \textbf{?`Se entienden los iconos y los símbolos de la apliDice que modificaría los de descarga de billetes y propone
    también añadir texto explicativo al pasar el ratón por encima. Pregunta Carlos que si cambiaría los de soporte y preguntas frecuentes
    y dice que se ven bien pero que a lo mejor los pondría junto al perfilcación?} Sí.
    \item \textbf{?`Se pueden hacer las acciones importantes con pocas pulsaciones?} Sí.
    \item \textbf{?`Es cómodo accionar los controles?} Sí.
    \item \textbf{?`Los flujos de aplicación son fáciles de entender?} Sí.
    \item \textbf{?`Los usuarios siempre encuentran los controles que necesitan?} Sí.
\end{itemize}

Los datos recogidos se han plasmado en una tabla\footnote{Accesible a través de:
 \url{https://docs.google.com/spreadsheets/d/1cQoEyahiQpcvWYoxbBLHAQGNvvUiymUFSWfQCr3xSyQ/edit?usp=drive_link}}.

\subsection[Alejandro]{Alejandro\footnote{Vídeo con la entrevista a Alejandro: \url{https://drive.google.com/file/d/1T_cVtO2g1prdwi1HEYWkfylEHocXM5ic/view?usp=drive_link}}}

La siguiente entrevista fue realizado por Pablo a Alejandro, amigo de Pablo, el día 4 de enero de 2024. La entrevista
comienza con Pablo agradeciendo a Alejandro su colaboración en la evaluación y pidiéndole permiso para realizar la grabación. Alejandro
acepta, se le explica el objetivo de nuestra aplicación y comienza el estudio, con los siguientes resultados en las tareas:

\begin{enumerate}
    \item No tiene problemas registrándose.
    \item Tampoco tiene problemas cerrando sesión.
    \item Puede iniciar sesión también sin problemas.
    \item Accede fácilmente al apartado de modificar perfil, y tarda unos pocos segundos en encontrar el DNI una vez dentro pero
        tampoco le supone ningún problema la tarea.
    \item Vuelve fácilmente a la página de inicio.
    \item No tiene problemas al realizar la búsqueda.
    \item Como en anteriores estudios, Alejandro busca inicialmente el filtrado por transporte en la zona de filtros. Pero rápidamente ve que
        está fuera y lo acciona.
    \item Pulsa en la selección del viaje en repetidas ocasiones, pero al ver que solo selecciona se
        da cuenta del botón de información y lo pulsa.
    \item Ningún problema en esta tarea.
    \item Tampoco tiene problemas realizando la ordenación.
    \item No tiene problemas.
    \item En esta tarea, el usuario comenta cuando está rellenando los datos de los pasajeros que no sabe muy bien si ya ha rellenado ambos. A pesar
        de esto termina bien la tarea.
    \item Consigue hacerlo sin problemas.
    \item Tampoco tiene problemas usando el chat de la atención al cliente.
    \item No tiene problemas para acceder a la sección de mis reservas y una vez dentro visualiza los datos con el botón de información y accede a la
        descarga con el botón junto a las tarjetas de manera adecuada.
    \item Una vez en mis reservas le cuesta encotrar el icono, pero tampoco tarda mucho, apenas unos segundos.
    \item Ésta es la tarea que más le cuesta hasta el momento. Comienza entrando en información del viaje y luego va a modificar viaje, por si se puede cancelar
        el viaje en alguna de esas. Comenta que había visto el botón de cancelar, por lo que podemos suponer que no sabía si cancelaba todos los viajes. finalmente
        lo pulsa y consigue terminar la tarea.
    \item Al cancelar confunde los botones de cancelar con descargar billete y tiene que decirle Pablo que es el de debajo.
\end{enumerate}

A continuación se hacen a Alejandro las preguntas que habíamos definido anteriormente para comprobar su opinión sobre nuestra aplicación:

\begin{itemize}
    \item \textbf{?`Sabes qué elementos son accionables?} Sí. 
    \item \textbf{?`Es fácil completar el proceso de registro?} Sí. 
    \item \textbf{?`Has conseguido encontrar el contenido que buscabas?} Sí.
    \item \textbf{?`Cambias fácilmente de una pantalla a otra?} Sí.
    \item \textbf{?`Se entienden los iconos y los símbolos de la aplicación?} Sí, pero dice que quitaría en los filtros las
        barras (que usamos para marcar disponibilidad) y pondría simplemente una línea con los números más grandes. 
    \item \textbf{?`Se pueden hacer las acciones importantes con pocas pulsaciones?} Sí.
    \item \textbf{?`Es cómodo accionar los controles?} Sí.
    \item \textbf{?`Los flujos de aplicación son fáciles de entender?} Comenta lo que habíamos puesto antes, de que al cancelar la reserva
        de un pasajero no le parece intuitivo porque parece que va a cancelar el billete entero.
    \item \textbf{?`Los usuarios siempre encuentran los controles que necesitan?} Dice que el icono para la descarga
        de los billetes no le parece intuitiva.
\end{itemize}

Los datos recogidos se han plasmado en una tabla\footnote{Accesible a través de:
 \url{https://docs.google.com/spreadsheets/d/1Y1xydtw9MEIXRZTJoLbDFNBL08jTwlq2Hr7sLkw_BPQ/edit?usp=drive_link}}.

\subsection[Silvia]{Silvia\footnote{Vídeo con la entrevista a Silvia: \url{https://drive.google.com/file/d/1mXlj10G3LwLzKnFJqhHM7GF2Bx-LTVdb/view?usp=drive_link}}}

La última entrevista se realizó el día 4 de enero a Silvia, siendo los moderadores Pablo y Javier. La entrevista comienza con Pablo agradeciendo a Silvia su participación en esta sesión de evaluación y explicando en qué va a consistir su participación. Le pide permiso para grabar esta sesión y comienzan con las tareas.

\begin{enumerate}
    \item Silvia realiza la tarea rápidamente y sin mayor problema.
    \item Al igual que la tarea, se hace de manera rápida y sin problemas
    \item Inicia sesión sin problemas.
    \item Encuentra de manera rápida la opción de modificación del perfil.
    \item Hace la búsqueda rápido y sin problemas.
    \item No presenta problemas tampoco en esta tarea.
    \item Silvia selecciona los vuelos sin problemas, al deshacer el filtro lo intenta en la barra de  filtros.
    \item Selecciona el viaje en un primer momento, luego ve el icono de detalles.
    \item Hace y deshace el filtro sin problemas.
    \item Encuentra la opción de ordenado rápido.
    \item Encuentra la opción de accesibilidad buscando en la barra de filtros.
    \item Pablo explica un problema con el prototipo para poder continuar fácilmente. Por lo demás la tarea se lleva a cabo sin problemas.
    \item Silvia no encuentra de primeras el botón de preguntas frecuentes. Al segundo intento lo consigue.
    \item Silvia no escribe nada pero envía el mensaje.
    \item De primeras Silvia no consigue encontrar la opción de consultar reserva, pero tras salir de modificar reserva si lo encuentra.
    \item Silvia sigue en la pantalla de consulta, luego ya sale de ella y va a la ventana correcta.
    \item Silvia sale de la pantalla de reservas para luego volver a entrar y ya encontrar la pantalla de cancelación.
    \item Silvia cancela el viaje completo sin problemas.
\end{enumerate}

Tras estas tareas se le hacen las siguientes preguntas a Silvia: 

\begin{itemize}
    \item \textbf{?`Sabes qué elementos son accionables?} Alguna vez no, pero la mayoría de las veces sí. 
    \item \textbf{?`Es fácil completar el proceso de registro?} Sí. 
    \item \textbf{?`Has conseguido encontrar el contenido que buscabas?} Sí.
    \item \textbf{?`Cambias fácilmente de una pantalla a otra?} Sí.
    \item \textbf{?`Se entienden los iconos y los símbolos de la aplicación?} Sí, pero, al igual que Alejandro, dice que
        quitaría en los filtros las barras (que usamos para marcar disponibilidad) y pondría simplemente una línea con los números más grandes. 
    \item \textbf{?`Se pueden hacer las acciones importantes con pocas pulsaciones?} Sí.
    \item \textbf{?`Es cómodo accionar los controles?} Sí.
    \item \textbf{?`Los flujos de aplicación son fáciles de entender?} Sí.
    \item \textbf{?`Los usuarios siempre encuentran los controles que necesitan?} Sí.
\end{itemize}

Los datos recogidos se han plasmado en una tabla\footnote{Accesible a través de:
 \url{https://docs.google.com/spreadsheets/d/1b-Dq4FmMZR0A-RFgesuxFtxe0JwtIUeLhUb8SLToksQ/edit?usp=drive_link}}.