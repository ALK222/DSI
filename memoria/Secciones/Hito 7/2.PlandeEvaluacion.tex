\section{Plan de Evaluación}
En esta primera sección de este hito se realizarán aquellas tareas previas a la fase de evaluación con los usuarios, que son necesarias para poder reaizar una correcta
evaluación. Estas tareas, como ya hemos visto, pasan por la definición de conceptos necesarios, como el propósito y los objetivos de la evaluación hasta la definición de 
metodologías y procedimientos con los que se van a extraer y tratar los datos necesarios.

\subsection{Identificación del propósito y los objetivos de la evaluación}
El propósito de esta evaluación es conseguir obtener una valoración acerca de los problemas que se encuentren en la aplicación y que no hayan sido detectados aún, 
bien porque los expertos no los observaron o porque no existía una heurística clara en la que poder enmarcar este error, por lo que no se tuvo en cuenta. En cuanto
a los objetivos de la evaluación, tenemos los siguientes:
\begin{itemize}
    \item Durante la sesión de evaluación con expertos, algunos de ellos tuvieron problemas para poder identificar cómo reproducir una acción concreta, por lo que 
    queremos intentar comprobar si las mejoras que hemos realizado son suficientes para que todas puedan ser completadas de forma autónoma por el usuario.
    \item Buscar posibles errores (tanto de diseño como de coherencia) que no hayan podido ser encontrados por los expertos al realizar la evaluación heurística.
\end{itemize}

\subsection{Formular las preguntas de investigación}
El siguiente paso es formular las preguntas de investigación, a las que tendremos que dar una respuesta basada en evidencias al haber finalizado esta fase de investigación.
Las preguntas que hemos identificado son:
\begin{itemize}
    \item ?`Los usuarios saben qué elementos son accionables? 
    \item ?`Es fácil completar el proceso de registro? 
    \item ?`Consiguen los usuarios encontrar el contenido que buscan?
    \item ?`Los usuarios cambian fácilmente de una pantalla a otra?
    \item ?`Se entienden los iconos y los símbolos de la aplicación? 
    \item ?`Se pueden hacer las acciones importantes con pocas pulsaciones?
    \item ?`Es cómodo accionar los controles?
    \item ?`Los flujos de aplicación son fáciles de entender?
    \item ?`Los usuarios siempre encuentran los controles que necesitan?
\end{itemize}

\subsection{Identificar los requisitos para los participantes}
Los participantes de nuestra evaluación deben cumplir una serie de requisitos mínimos para poder participar en la sesión. Estos son lo siguientes:
\begin{itemize}
    \item El principal, aunque puede ser evidente, es que tengan la capacidad de leer y escribir. Aunque la aplicación tiene una serie de símbolos e iconos para 
    facilitar la comprensión, para ciertas funcionalidades como puede ser la selección del origen y destino o la selección de servicios adicionales, esta capacidad 
    es indispensable.
    \item Buscaremos usuarios tanto que tengan experiencia con otros comparadores de viajes como otros que no tengan ninguna (pero preferentemente que deseen 
    usarlo, para que coincida con un usuario potencial). De esta manera podremos evaluar la consistencia externa de la aplicación (en el caso del primer grupo), 
    así como lo intuitiva y accesible para nuevos usuarios que es nuestra página (segundo grupo).
    \item En la medida de lo posible, intentaremos hacer la evaluación con alguna persona que tenga discapacidad, para así poder estudiar las funcionalidades 
    añadidas para este grupo.    
\end{itemize}

\subsection{Descripción del diseño experimental}
Just Travel It es una aplicación web de viajes que permite a los usuarios poder buscar entre una gran cantidad de viajes y ofertas. Dentro de la página del 
comparador puedes filtrar todas las opciones que te aparecen para encontrar la que más se adapte a tus necesidades.

\subsubsection{Listado de tareas}
Veamos el listado de tareas que tiene que realizar el usuario durante la evaluación:
\begin{enumerate}
    \item Registrarse
    \item Cerrar sesión
    \item Iniciar sesión
    \item Entrar al perfil y modificar el DNI del usuario.
    \item Volver a la página de inicio desde el perfil del usuario.
    \item Buscar un viaje de Madrid a Barcelona desde el 16 de diciembre de 2023 hasta el 22 del mismo mes y para dos viajeros.
    \item Filtrar los viajes por el tipo de transporte solo de avión y deshacer el filtro.
    \item Ver los detalles del viaje de ida.
    \item Filtrar los viajes por un precio menor o igual a 25 euros y deshacer el filtro.
    \item Ordenar los resultados del comparador por precio de menor a mayor.
    \item Filtrar los viajes accesibles y volver a viajes no accesibles.
    \item Seleccionar un viaje de ida y vuelta y seleccionar los asientos para cada pasajero y pagarlo con tarjeta.
    \item Consultar en preguntas frecuentes ¿Cómo modificar mi reserva?
    \item Solicitar ayuda en el soporte electrónico del chat escribiendo “Hola”.
    \item Consultar los datos de una reserva en el perfil de usuario y descargar los billetes.
    \item Modificar los servicios adicionales del primer viajero de una reserva.
    \item Cancelar el viaje del segundo pasajero.
    \item Cancelar el viaje completo.    
\end{enumerate}

\subsubsection{Descripción del entorno y las herramientas que se van a utilizar}
Las sesiones se realizarán vía videollamada por Google Meet (para poder compartir la pantalla y poder ver qué hace el usuario en todo momento de la sesión) y 
serán grabadas (con previo consentimiento del usuario final). El usuario recibirá un enlace al prototipo de Figma, donde podrá realizar la prueba del funcionamiento 
de la aplicación. Para poder realizar esta sesión será necesario un ordenador con cámara integrada (para poder grabar el comportamiento del usuario), con conexión 
a Internet para acceder al prototipo. Asimismo, se necesitará un cronómetro para medir los tiempos que tarda el usuario en completar las tareas.

\subsubsection{Tareas que va a realizar el moderador}
Idealmente, en todas las sesiones de evaluación va a haber un total de dos interventores. Uno de ellos actuará en calidad de moderador, realizando la sesión 
y guiando al usuario en las tareas que tiene que realizar y en caso de que tenga alguna duda poder solventarla. Por otro lado, la otra persona se encontrará 
observando el transcurso de la sesión, anotando las consideraciones necesarias (rellenando la plantilla de datos que se van a recolectar) e interviniendo en 
caso de que sea necesario.

\subsubsection{Identificación de los datos que se van a recolectar}

\subsubsection{Descripción de la metodología de análisis de datos}