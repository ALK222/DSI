\documentclass[twoside]{AiTeX}


\title{Memoria DSI}
\author{Grupo 2}
\date{Octubre 2023}
\begin{document}
%\datos{facultad}{universidad}{grado}{asignatura}{subtitulo}{autor}{curso}
\datos{Informática}{Universidad Complutense de Madrid}{Ingeniería informática}{NOMBRE APLICACION}{Memoria de proyecto - Hito 1}{Alejandro Barrachina Argudo \\
Maria Esteban Benito \\
Carlos Varela Sansano \\
Javier Gómez Arribas \\
Guillermo Novillo Díaz \\
Leire Jiménez González \\
Laura Martínez Tomás \\
Sergio Colet García\\
Daniel Yllana \\ 
Pablo Lavandeira Poyato \\
Rodrigo Souto Santos 
}{2023-2024}
\portadaApuntes
\pagestyle{empty}
\tableofcontents 
\pagestyle{empty}
\justify
\pagestyle{fancy}

\newpage

\newacronym{di}{D.I.}{Discapacidad Intelectual}


\section*{Control de cambios} %
\noindent\begin{tabularx}{\textwidth}{ |l|l|p{5cm}|X| }
    \hline
    \textbf{Versión} & \textbf{Fecha} & \textbf{Autores}     & \textbf{Descripción}                                                 \\
    \hline
    1.0              & 27/09/2023     & Alejandro Barrachina & Comienzo de la memoria\\
    \hline
\end{tabularx}

\newpage

\chapterA{Introducción}

\textit{\app} ha surgido como proyecto debido a la necesidad de viajar que tienen las distintas personas con \gls{di}. Al
descubrir que muchas de éstas tienen problemas a la hora de reservar sus viajes con las páginas actualmente disponibles, decidimos crear un 
comparador de viajes que sean capaces de usar las personas con una limitación intelectual leve, o sus acompañantes en el caso de personas que
tengan una discapacidad mayor. 

Para lograr esto, vamos a empezar realizando una investigación sobre los usuarios potenciales y sus necesidades. Para esto realizaremos entrevistas
a distintos perfiles dentro de nuestras hipótesis de personas, para así poder diseñar la aplicación en referencia a sus experiencias y problemas a la hora
de viajar y/o buscar viajes. Junto a las entrevistas, también se realizará la observación de los usuarios en ámbitos de búsqueda de viajes, para poder hacer
una observación de éste. Además, para alcanzar una mayor demografía, haremos uso de cuestionarios, los cuáles nos permitirán también recibir información
en menor medida que en una entrevista, pero de más gente.

Después de la investigación, realizaremos un modelado de la aplicación, utilizando los resultados obtenidos para crear arquetipos de usuarios que
contengan información sobre los distintos objetivos que tendrían los perfiles potenciales. Gracias a esto, podríamos obtener usuarios que nos den
\textit{feedback} a la hora de avanzar con el diseño del sistema.

\chapterA{Plan de investigación}

\chapterA{Investigación}

\section{Introducción}

Para poder diseñar una aplicación correctamente, es muy importante realizar previamente una investigación para saber qué
es lo que realmente se necesita, y cuáles son los problemas de nuestro público objetivo. Hay muchas maneras de conseguir esto, pero en nuestro
caso, como desgraciadamente no disponemos del tiempo para poder usar todos los métodos, hemos realizado las siguientes.

\begin{itemize}
    \item \textbf{Entrevistas:} Es una de las partes más importantes de la investigación. \textbf{CONTINUAR}
    \item \textbf{ESTO CREO QUE MEJOR HACERLO AL FINAL PARA VER LO QUE TENEMOS HECHO}
\end{itemize}



Pero antes de realizar esta labor, debemos saber reconocer cuál es el público objetivo de nuestra aplicación y
de qué manera podemos clasificar a los distintos perfiles dentro de los clientes potenciales. Para eso hemos realizado la
\textbf{Hipótesis de personas}.

\section{Hipótesis de personas}

En esta primera fase de investigación, el primer paso que vamos a seguir es la identificación de los posibles usuarios que vamos a tener
en nuestra aplicación. Nuestro principal objetivo, como hemos visto anteriormente, es ofrecer una herramienta que permita a las personas con
discapacidad intelectual tener la posibilidad de utilizar un comparador de viajes sin problemas. Los principales usuarios que hemos identificado
y los cuáles vamos a entrevistar en la posterior fase de entrevistas son los siguientes:

\begin{itemize}
    \item \textbf{Personas con \gls{di}:} Obviamente, este es nuestro público principal. Son las personas en las que pensamos cuando
            elegimos diseñar esta aplicación. Pero dentro de esta categoría podemos subcategorizar a los individuos:
            \begin{itemize}
                \item \textbf{\gls{di} leve o moderado:} La mayoría de personas con \gls{di} pertenecen a este grupo, y son en las que más nos vamos a centrar.
                        Esto es debido a que en su mayoría tienen un mayor nivel de independencia, por lo que será más fácil que quieran hacer algún viaje, ya
                        sea en solitario o en compañía.
                \item \textbf{\gls{di} grave:} En estos casos, deberíamos considerar la dependencia de la persona. Es menos probable que estos individuos
                        se metan en la página, ya que por lo general disponen de un tutor o alguien a cargo que será el que lo organice en caso de viajer. En
                        este caso, el tutor sería el cliente potencial, ya que es el que usaría la aplicación. 
            \end{itemize}
    \item \textbf{Acompañantes de personas con \gls{di}:} También es un público importante de nuestra aplicación. Ésto es debido a que en muchas casos las personas con
            D.I. será tan dependiente que necesitará de otra persona para poder realizar la búsqueda. Éstas personas pueden enfrentarse a distintos problemas
            a la hora de reser var que habrá que tener en cuenta.
\end{itemize}

Por otro lado, vamos a tener otros tipos de personas identificados, que no van a ser usuarios potenciales de nuestra aplicación pese a pertenecer a 
alguno de los anteriores grupos, por lo que en el momento que detectemos que se trata de una persona encuadrada en uno de estos tipos
vamos a finalizar la entrevista ó el cuestionario, ya que no vamos a poder extraer información de valor para nuestra aplicación.


\begin{itemize}
    \item {\textbf{Usuarios que prefieren viajar con todo planificado por una agencia:}} Se trata de aquellos usuarios que cada vez que quieren
        reservar un viaje no les importa realizar un gasto extra y prefieren que todo sea organizado por una agencia de viajes, sobre todo de cara a
        evitar la aparición de ciertos conflictos que pueden tener con otras aplicaciones de la competencia.
    \item {\textbf{Usuarios que no les guste viajar y no tengan la necesidad:}} Existen usuarios que no les gusta viajar y que además nunca se han visto
        (ni se van a ver en un futuro próximo), por lo que no nos van a resultar de interés para la aplicación, ya que el objetivo buscado son perfiles que
        hayan experimentado el proceso y puedan contarnos aquellos inconvenientes que han podido encontrarse a lo largo del proceso.
\end{itemize}
 
A modo de conclusión, los perfiles de usuario que tenemos, como se puede apreciar está influenciado por dos factores muy importantes: el grado de discapacidad de la persona
y el grado de interés que tiene en planificar sus propios viajes. Estas cualidades van a ser necesarias para poder identificar el tipo de usuario frente al que nos encontramos,
por lo que será el eje de muchas de las preguntas presentes tanto en las entrevistas como en los cuestionarios.


%%%%%%%%%%%%%%%%%%%%%%%%%%%%%%%%%%%%%%%%%%%%%%%%%%%%%%%%%%%%%%
\section{Entrevistas}
%%%%%%%%%%%%%%%%%%%%%%%%%%%%%%%%%%%%%%%%%%%%%%%%%%%%%%%%%%%%%%

Es la parte más importante de la investigación, ya que es de donde conseguiremos obtener más información. Consiste en realizar una serie de preguntas al usuario para
ver si encaja con los perfiles objetivo de nuestra aplicación, y en caso de hacerlo, obtener los datos necesarios para poder diseñarla. Gracias a esto podremos averiguar
cuáles son los problemas que tienen estas personas con los comparadores actuales y qué necesidades tendrían.

Tenemos distintos clientes que forman parte de nuestra hipótesis de personas, y ambos tienen necesidades distintas. Por tanto tendremos distintas preguntas
dependiendo del perfil al que nos enfrentemos.

Todas las entrevistas comenzarán presentándonos y preguntando el nombre al entrevistado. Tras eso, tendremos que pedir autorización para grabar imágenes, ya que las
grabaciones son necesarias para un posterior análisis y recabar así la mayor cantidad de información posible. Para que la entrevista sea más distendida, preguntaremos
si le podemos tutear. Tras esto, explicaremos nuestros objetivos con la aplicación y procederemos a realizar las preguntas.

\subsection{Preguntas a usuarios con discapacidad}

Al comienzo realizaremos el denominado como \textit{Screener}, es decir, una serie de preguntas para ver si el entrevistado es un cliente potencial. En caso de serlo,
podremos proceder con el resto de preguntas.

\begin{enumerate}
    \item {\textbf{?`Cuántos años tienes?:}} sirve para encuadrar al usuario dentro de un marco de edad concreto y poder tratar con él en función de esto.
    \item {\textbf{?`Qué tipo de discapacidad tiene?:}} sirve para luego adaptar las preguntas en función de sus discapacidad porque no va a ser lo
                mismo para alguien con discapacidad física que sensorial.
    \item {\textbf{?`En qué grado (leve, moderado, grave)?:}} sirve para adaptar más las preguntas de las entrevistas.
    \item {\textbf{?`Qué impacto tiene esta discapacidad en tu día a día?:}} sirve para adaptar más las preguntas de las entrevistas.
    \item {\textbf{?`Cómo de cómodo te sientes con la tecnología?}}
    \item {\textbf{?`Te gusta viajar? ?`Cuéntame por qué?:}} es la pregunta que nos va a determinar si el usuario es potencial de la aplicación
                o si bien lo tenemos que descartar.
    \begin{enumerate}
        \item {\textit{Sí:}} usuario potencial. Tenemos que conocer ahora si le gusta viajar por ocio o bien lo tiene que hacer por negocios.
        \item {\textit{No:}} puede seguir estando dentro de la hipótesis de usuarios. En este caso las preguntas a hacer van a variar y van a
                        depender de la respuesta que nos de.
    \end{enumerate}
    \item {\textbf{?`Suele viajar?:}} sirve para identificar si el usuario es apto, porque si no viaja no tiene sentido la aplicación.
    \item {\textbf{?`Le gustaría viajar más?:}} sirve para saber si el usuario que no viaja tiene pensado viajar en un futuro y por tanto, considerarlo apto.
    \item {\textbf{?`Disfruta cuando viaja?:}} sirve para entender al usuario que viaja y si va a usar la aplicación más o menos frecuente.
    \item {\textbf{?`Cuál es tu medio de transporte favorito y por qué?:}} sirve para entender al usuario que viaja y si va a usar la aplicación más o menos frecuente.
    \item {\textbf{?`Qué tipo de viajes has hecho?:}} sirve para entender al usuario que viaja y si va a usar la aplicación más o menos frecuente.
    \item {\textbf{?`Sueles necesitar acompañante para tus viajes?:}} tenemos que tener en cuenta si la persona con la que estamos tratando requiere de la
                ayuda de un acompañante que viaje con él para tenerlo en cuenta a la hora de desarrollar la aplicación y nos ayuda a conocer un poco al entrevistado.
    \item {\textbf{?`Por qué motivos suele viajar?:}} sirve para identificar las motivaciones del usuario.
    \item {\textbf{?`Qué es lo que más le dificulta a la hora de viajar? ?`Hay algo más que le dificulte viajar?:}} sirve para identificar molestias que tiene
                el usuario al planificar un viaje.
    \item {\textbf{?`Cuando vas a organizar un viaje, que es lo primero que haces?}}
    \item {\textbf{?`Te encargas tú de organizar el viaje?:}} queremos conocer si el usuario tiene la iniciativa para organizar el viaje por sí solo o bien si
                recurre a profesionales como agencias de viajes o a terceras personas.
    \item{ \textbf{?`Cómo has organizado tus viajes?:}}
    \begin{enumerate}
        \item {\textit{No:}} ?`No has pensado nunca en usar un comparador de viajes?: queremos conocer si aunque el usuario recurra a agencias de viajes o
                        a otras personas para realizar el viaje usaría en algún momento nuestra aplicación.
        \item {\textit{Si:}} ?`Te has encontrado alguna dificultad en el proceso?: está bien para finalizar el screener e introducir la siguiente parte.
    \end{enumerate}
    \item {\textbf{?`Te resulta más cómodo realizar estas búsquedas de viajes en una aplicación móvil o en una página web?:}}
    \item {\textbf{?`Cuál es el factor clave que hace que se decante por esa opción en un viaje?:}} sirve para saber sus prioridades.
    \item {\textbf{?`Influye el coste del viaje en su elección?:}} sirve para saber más información.
    \item {\textbf{?`Cuál de las partes de una página tradicional de comparación de viajes te parece más tediosa?:}} necesitamos conocer los problemas que
                puede encontrarse el usuario en las páginas tradicionales para tenerlo en cuenta y poder mejorarlo en nuestra aplicación. En caso de que
                la respuesta sea afirmativa, podemos preguntarle si existe alguna opción de ayuda dentro de la plataforma.
    \item {\textbf{En caso de que hayas tanido algún problema en estas páginas, ?`has podido solicitar ayuda de manera sencilla?:}}
    \item {\textbf{?`Según tu opinión, ?`cómo debería ser la forma ideal en la que una aplicación muestre la información?:}}
    \item {\textbf{?`Hay alguna función de las páginas tradicionales que consideres útil para tus necesidades?:}}
    \item {\textbf{En caso de que hayas realizado alguna reserva de viaje, ?`has conocido y sido informado de forma clara de las condiciones
                        y políticas de cancelación del viaje?:}}
    \item {\textbf{?`Te gustaría que estas páginas incluyesen más información sobre accesibilidad para viajeros con discapacidad?:}}
    \item {\textbf{?`Podrías darme algún ejemplo de aplicación que te guste y uses a diario?}}: queremos poner al usuario en una situación
    en la que nos comente una aplicación que le guste para poder conocer los motivos que le llevan a ello.
    \item {\textbf{?`Consideras que es una aplicación accesible??`Por qué?}}: queremos conocer desde el punto de vista de la persona aquellos
    elementos y problemas que puede identificar dentro de la aplicación y que puedan suponer un problema.
    \item {\textbf{Acordarse de hacer el debriefing, para hacer un repaso de lo que ha dicho a ver si se acuerda de algo más.}}
    \item {\textbf{?`Se te ocurre algo más de lo que hemos hablado que podría ayudarnos?}}
\end{enumerate}

Al finalizar, le agradeceremos al entrevistado su tiempo y su participación en nuestro proyecto.


\section{Resumenes de entrevistas}

\section{Lista de factoides}

\section{Analisis de la competencia}

\section{Mapas de empatía}


\printglossary[title={Glosario}]

% DESCOMENTAR SI SE USAN IMÁGENES
% \let\cleardoublepage\clearpage
% \listoffigures
% \addcontentsline{toc}{chapter}{Índice de figuras}
% \let\cleardoublepage\clearpage

% DESCOMENTAR SI SE USAN TABLAS
% \listoftables
% \addcontentsline{toc}{chapter}{Índice de cuadros}
\end{document}
