\documentclass[twoside]{AiTeX}



\title{Memoria DSI}
\author{Grupo 2}
\date{Octubre 2023}
\begin{document}
%\datos{facultad}{universidad}{grado}{asignatura}{subtitulo}{autor}{curso}
\datos{Informática}{Universidad Complutense de Madrid}{Ingeniería informática}{NOMBRE APLICACION}{Memoria de proyecto}{Alejandro Barrachina Argudo \\
Maria Esteban \\
Carlos Varela Sansano \\
Javier Gómez Arribas \\
Guillermo Novillo Díaz \\
Leire Jiménez González \\
Laura Martínez Tomás \\
Sergio Colet García\\
Daniel Yllana \\ 
Pablo Lavandeira Poyato
}{2023-2024}
\portadaApuntes
\pagestyle{empty}
\tableofcontents
\pagestyle{empty}
\justify
\pagestyle{fancy}

\newpage

\section*{Control de cambios} %
\noindent\begin{tabularx}{\textwidth}{ |l|l|p{5cm}|X| }
    \hline
    \textbf{Versión} & \textbf{Fecha} & \textbf{Autores}     & \textbf{Descripción}                                                 \\
    \hline
    1.0              & 27/09/2023     & Alejandro Barrachina & Comienzo de la memoria\\
    \hline
\end{tabularx}

\newpage

\chapterA{Descripción de la aplicación}

\printglossary[title={Glosario}]

% DESCOMENTAR SI SE USAN IMÁGENES
% \let\cleardoublepage\clearpage
% \listoffigures
% \addcontentsline{toc}{chapter}{Índice de figuras}
% \let\cleardoublepage\clearpage

% DESCOMENTAR SI SE USAN TABLAS
% \listoftables
% \addcontentsline{toc}{chapter}{Índice de cuadros}
\end{document}
